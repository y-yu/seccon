さて敵チームに這い寄られて精神力が低下する前に、
こちらがさらに点差を広げて敵の精神力を破壊したい所だ。
とりあえず先ほどメールヘッダ汚染を仕掛けた\Marsだが、
メールはまだ届いていない。
これは攻撃が失敗したと考えるしかあるまい。
となると、やはりメールヘッダ汚染は不可能なのであろうか。

「吉村君」

薄井くん\footnote{情報科学類二年次、ガチチャリ勢。}%
に話しかけられた。
彼もめいす同様この全国大会に合わせて追加したメンバーで、とりあえずLISP信者。

「今ポートスキャンしたんだけど、なんか22番ポートが開いてる」

22番といえばSSHだ。

「パスワード認証みたい」

聞き耳を立てていたおーぴーが追加する。
これは……メールヘッダ汚染の可能性が出てきたぞ。
今までの状況では仮にメールヘッダ汚染が成功したとしても、
その先どうやってFLAGに書き込めばいいのか見当がつかなかったが、
SSHがあるのであれば話は別だ。
つまり送信されるメールにSSHのユーザーとパスワードが付いていて、
そのメールをヘッダ汚染でこちらに引っ張ることでSSHにログイン出来るようになる。
そしてSSHでログインしてからは何か別の障壁が用意されている、というパターンだ。

だが先ほどの攻撃は失敗した。
これはメールヘッダ汚染の失敗とも考えられるが、
どちらかと言えば競技という性質上の問題ではないかという推論も出来る。
というのも競技故に、問題の置いてあるサーバー、つまりは\Marsを外部へ接続する訳にはいかなった。
しかしこの\ctt{message.cgi}にはメールヘッダ汚染の脆弱性があり、かつ\MarsはLANには接続されている。
従って、どこかLAN内にメールサーバーを設置すればそこにメールが届くという可能性がありえる。
ただ問題はある。
今サーバーのIPアドレスは\url{10.0.2.5}だが、
俺に割り振られたIPアドレスは\url{192.168.7.4}だ。
ネットワークのことはよく分からないが、こういう構成で本当に到達出来るのだろうか。
それに俺はメールサーバーを構築した経験などないので、
どのようにすればいいのかさっぱり分からない。

仕方ない……。
一旦この手法は見送ろう。
明日までにメールサーバーを用意するしかあるまい。
メールのような一般的なシステムであれば、たぶんCPANなど様々な選択肢があるだろう。

