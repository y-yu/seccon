ひたすら\upkeyと\keytop{Enter}キーを交互に連打。
俺が妨害工作をせっせとしている間、
おーぴー達が\MercuryのFLAGへ書き込むために何か手を打っているはずだ。
FLAGは先程\ctt{gomi.php}で調べたところ、\ctt{stage5}なる権限でのみ書き込めるようだった。
これはつくば大会であったような、
\ctt{sudoers}を読み取るようなパターンが想像されるが、
今俺がそれを調べることは出来ない。
今は大量にアップロードされるゴミファイルを駆逐せねばならない。

おーぴー達の進捗が分からぬままひたすら\ctt{rm}を連打していた時、
会場にサイレン音が響く。
キーワードを取得した時にもサイレンが鳴るのが、
これはそれより長いサイレン音。
スコアサーバーには、
\urandomが\MercuryのFLAGにフラッグワードを書き込んだことを示す表示が点灯している。

よし。
流石はおーぴー。
ベガスで戦っただけあって、強い。

そして、めいすが書いていたシェルスクリプトも完成した。
\ctt{sleep}と\ctt{rm}を無限ループする単純なプログラムだが、
相手からしたらそう簡単には防げまい。
そうだ、\ctt{authorized\_keys}を確認しておくか。

\end{multicols}

\lstinputlisting{src/2/authorized_keys2}

\begin{multicols}{2}

Fuck!明らかに増えてる……!
やはり俺の妨害工作をする前に、
一定数の人間が\ctt{authorized\_keys}まで到達してしまったのか。
いや、妨害工作とはいえ先程までは俺が手動で\ctt{rm}していたに過ぎん。
俺が削除する間を突いてスクリプトを実行したという可能性もある。

「吉村くん」

おーぴーから話しかけられる。

「おーぴー、\ctt{authorized\_keys}が増えてる。
どっかのチームが捩じ込んでる」

「ログインしている敵チームの人を\ctt{kill}するから、\ctt{\$BASHPID}を教えて」

「ええと、どうすれば?」

「\ctt{echo \$BASHPID}でいい」

なるほど、既に策は考えてあるのか。
ならば直ちに。

\begin{lstlisting}
[stage4@localhost .ssh]$ echo $BASHPID
2011
\end{lstlisting}

「俺の\ctt{\$BASHPID}は\ctt{2011}だ」

「了解」

おーぴーはメンバー全員の\ctt{\$BASHPID}を集め、
\ctt{pkill}でそれら以外のプロセスを殺し始める。
まあよく分からないが、たぶんこれで我々以外の人間が\ctt{authorized\_keys}に鍵を登録出来たとしても、
SSHでログインした瞬間に殺されてしまうんだろう。

% vimexec: let g:tex_no_math = 0
% vimexec: syntax off
% vimexec: syntax on
% vimexec: source ~/.gvimrc
