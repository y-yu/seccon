\Mercuryは安定運用の次期に入った。
このフラッグワードというものは一度書き込むと、
ポイントが一発入って終わりというものではないらしい。
ゆにゃによると、5分ごとに変化するフラッグワードを変わる度にFLAGへ書き込むことで20ボイントになるそうだ。
キーワードのコミットが100ポイント程度なので、20分でキーワード一個に相当することになる。

今のところCTFの問題が\Mercuryしかないのも\urandomにとってはありがたい。
FLAGを取得しているのは\urandomしかいないのだから、
他のチームも\urandomが妨害工作をしていると薄々気づいているかもしれない。
しかし、妨害をしていると知っていても他に解く問題がないのだから仕方ない。
それに妨害していると知っているのは妨害をしている我々だけだ。
他のチームにとっては、アップローダーが正しく動作しないのが、
あるいはSSHに接続すると瞬時に切断されるのが単なるCTFの仕掛けなのか、
それとも敵チームの工作なのか判断出来ない。
つくば大会などでは問題を解く度に加点というシステムであったので、
序盤多少出遅れた所で最後に点を取りまくれば逆転もありえた。
しかし全国大会のシステムでは%FLAGが無人の城に放置されている。
FLAGを取ればまず城下町から税金を5分ごとに搾取出来、
さらに城の設備を使って敵チームからFLAGを防衛出来る。
この一石二鳥システムによって敵チームはフラッグが取れず、
こちらは税金で単調増加。

「あれ?このアップローダー、消えるんだけど」

\EDの島から声が飛んできた声を盗み聞く。
\EDも多分、本当は俺なんかが太刀打ち出来るような相手じゃないのだろう。
しかし、彼らは多分初動をしくじった。
ネットワーク機器の故障とか、コンピュータの設定ミスとか、
あるいはJonh The Ripperを持っていなかったとか。
理由は様々考えられるが、とりあえず彼らは最初のあたりで躓いた。
それが致命的で、\Mercuryはもはや俺達が侵入した時とは難易度が遥かに異なる別の物体になった。
こんな簡単に、優勝候補が堕ちるなんて。
俺達なんて、新聞に一文字たりとも載らなかったのに。
%JOI\footnote{情報オリンピックのこと。}にも未踏\footnote{情報処理推進機構が実施する未踏IT人材発掘・育成事業のこと。}%
%にも通ってない、平均点を這うように生きている俺みたいな人間を擁するチームに\EDが遅れをとったなんて。

大会が始まって数時間、
俺が掲げた目標が完了してしまった。
なんか、あっけないな。
天皇人間宣言みたいに、今まで神だと思っていたものが、
突然人間になったみたいにあっけない。
などと油断すると背後から刺されるのかもしれないが。

