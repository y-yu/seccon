編集部でメールサーバーの用意を済ませて帰宅する。
真円の月光で明るい夜を自転車で走りながら、
そろそろ抗ヒスタミン剤を用意せねばならんなどと思う。

薬といえば、今日はハルシオン\footnote{睡眠導入剤のこと。}%
を入れようか迷うな。
ハルシオンはナイフみたいに切れ味のある鋭利な睡眠導入剤だと思うけど、
時々サイドエフェクトなのか次の日が鉛みたいに眠い時がある。
大学の授業であれば眠くても問題ないのでそのまま適当に服用するのだけど、
明日は全国大会の二日目なので、そういうわけにもいかない。
ただ、あの眠気は前に使っていたエビリファイ\footnote{抗不安薬・向精神薬として使われる薬のこと。}%
の副作用なのかもしれない。
いずれにしても最悪の場合、アラームでも起床出来ぬ程に昏睡してそのまま二日目を寝過ごすという可能性もある故、
今日はやはり入れない方がいいかもしれないな。
しかしこういう肝心な時に使えないというか、
使用を躊躇させるような副作用があるハルシオンはちょっと危ないな。
最近はロゼレム\footnote{睡眠導入剤のこと。}を貰っているが、
これは正直睡眠導入剤としてはイマイチで、
先生が言っていたように体内リズムを改善するサプリメントみたいな感じだ。
やはり今度はマイスリー\footnote{睡眠導入剤のこと。}あたりに挑戦してもいいかもしれない。

脳内会議で今日のハルシオン使用を止める判断を下した頃に家へ到着した。
適当にシャワーを浴びるなどして寝るための支度を整える。
そして、ハルシオンは入れずロゼレムとセディール\footnote{抗不安薬のこと。}だけ服用して布団に入る。
携帯電話のアラームを設定して寝ようとする。
だいたい思った通りだけど、こういう重要な予定があるような時というのは寝られない。
どうしようかな、ハルシオン、やっぱり入れとこうかな。
だけどハルシオンは変な時間にいれると体内リズムが破壊されるような印象があるから、
入れるならなるべく早いうちに決断したい……。
こういう時は神の決断を仰ぐしかあるまい。
今時間を確認して「分」が偶数か奇数かで入れるか入れないかを決定しよう。

よし、決した。
偶数であれば入れる、奇数であれば入れず。
脇に置いてある携帯電話の電源ボタンを押して、時間をチェックする。
結果は0時14分、偶数によりハルシオンの導入が決定された。
神の仰せの通りに俺はハルシオンを追加で飲み、眠ることにした。
仮に明日眠気が残ったとしても、会場にたくさん用意してあったエナジードリンクでかき消せばよい。
ハルシオンとカフェンインのどちらが強いか、確かめるいい機会となるだろう。
