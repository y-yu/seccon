結局\Marsはキーワードすら取ることが出来ぬまま一日目の競技が終了した。
その後東京電気大学内にある食堂と思しき施設にて立食会が催された。
つくば大会の時もそうであったが、
無料で参加しているにも関わらずこのように優遇されるというのは大変ありがたい。
ただつくば大会の時と大きく違うのは参加している人間の質だ。
つくば大会の時は競技者と運営が主であったものの、今回は圧倒的に記者や企業の人間が多い。

メンバーとオードブルの肉などを突きながら周囲の音声に耳を傾ける。
\urandomは結局800点差程度で二位の\ZxZや\wasamusumeなどを突き放し単独一位だ。
終盤に\Mercuryの防衛が崩れたが、まあもはやそれほど問題でもなかろう。
今回の全国大会は全体的に問題が難しいように思えるから、
このまま誰も解けない問題が大量に残る形で終焉を迎える公算が高い。
またつくば大会のようなシステムではないから、
問題に関するファイルを持ち帰って徹夜で解くという戦略も難しいはず。

おーぴーは我々と離れた所で誰かと話している。
今我々は箱根駅伝で言えば往路優勝したチーム。
優勝候補でもなんでもないようなチームが序盤に飛び出して一挙に大量リード、
この貯金でそのまま総合優勝という可能性も十分ありえるのだから、
\urandomが今は優勝候補だ。
そしておーぴーは優勝候補の中の筆頭故、記者か何かの取材を受けていてもおかしくはなかろう。

薄井くんが\EDの誰かと思われる人間と話している会話を盗み聞きしながら、
筑波勢である\ifconfigの人達と話す。%くりす\footnote{情報科学類三年次、ふぁぼガチ勢。}さんなどと話す。
話すにあたって、もはや明らかなのかもしれないが\urandomが妨害工作をやっていることと、
その詳細な手口については秘密とおーぴーに言われたので、一応その事に気を配る。
\Marsの話を何か聞き出せないかとも思ったが、そんな簡単に情報など出るわけもなかった。

一通り喰った後に会場を後にし、
帰りのTX車内で明日に向けたミーティングをする。
俺は\Marsに関することを一通り話して終った。
