俺とおしろ\footnote{情報科学類二年次、WORD民。}はTXつくば駅の改札前でおーぴー達と合流した。
今回の会場は北千住にある東京電気大学のキャンバスらしいので、
我々はTXで四十分程かけて向うことになる。
おーぴーはタイヤの付いたキャリーバッグと、
23インチくらいのモニターを装備していた。

列車に乗り込み、俺はKindleで本を読みながらも今日の大会のことをダラダラと考えている。
今日の全国大会はつくば大会などで行われたCTFとはシステムが違う。
つくばなどで行われた大会は、いくつか問題が用意されていて、
その問題に解答する度にポイントが入るという本当に\ruby{旗とりゲーム}{Capture\hfil The\hfil Flag}であったが、
今回行われるのは攻防戦。
\DEFCON\footnote{ラスベガスで行われる国際的なCTF。}%
などと同じように、
各チームにサーバーが与えられ、
敵チームのサーバー上で稼動しているサービスを攻撃しつつ、
自身のチームのサーバー上で動くサービスを守り抜くというようなものになるだろうというのがおーぴーの読みだ。
ちなみに大会実行委員会から提示されたルールは、

\begin{itemize}
	\item 指定のターゲットサーバを攻略し、また他チームの攻撃から守る攻防戦を行う。
	\item ターゲットサーバを攻略し得られる「キーワード」を、スコアサーバにサブミットしたチームに、点数を与える。
	\item ターゲットサーバを管理していることを示す「フラッグワード」を、所定のWebページに示していた場合、そのチームに点数を与える。
\end{itemize}

という3つのみで、正直何が何やら分からない。
まあ、恐らく何が何やら分かっていないのは他チームも同じだろうから、
こちらは\DEFCON参加者であるおーぴーを擁している分他チームよりも有利だろう。
いずれにしろつくば大会ではただ立っている旗を取れば良いだけであったが、
今回は取った旗を防御する必要あろうというのはなんとなく想像がつく。

%自分達の旗を守り、かつ敵の旗を倒す。
%さて、どうなることやら。
