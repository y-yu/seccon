北千住駅の近くで適当に昼食を済ませ、会場である東京電機大学の百周年記念ホールなる場所へ行く。
キャンパスは新しいということだけあって、筑波大学の中でも綺麗な方である総B棟などと勝負しても、
たぶん勝てぬだろうというくらいには綺麗な建物であった。

会場は広めのホールに机と椅子を用いて各チームの島が作られており、
もう場所は決められていた。
おーぴーなどはモニターを持ち込んでいることもあり、いそいそと設営を始める。
俺はMacBook Pro一台で勝負に臨む予定だったので、直ちに設営が終わり、
その間、入場の時に貰った新聞を眺めることにした。
新聞というのは二月二十二日の読売新聞夕刊で、ここに今回の\SECCON全国大会の記事が掲載された。

「学生ハッカー 実線で競う」と題された記事\footnote{\url{http://www.yomiuri.co.jp/kyoiku/news/20130222-OYT8T01153.htm}}には、
灘中学・高校のパソコン研究部のメンバーを中心としたチームが優勝候補であると書かれている。
これはおーぴーが警戒していたチーム\ED。
灘中学・高校という凄まじい学歴もそうだが、奈良大会\footnote{つくば大会の次に行われた\SECCON大会。}
においては準優勝であり大学院生で構成されたチーム\itokagiと二倍程度の点差を付けて優勝した。
また\EDは全員セプキャン\footnote{情報処理推進機構(IPA)が主催するセキュリティーイベント。えりっくなどが参加している。}卒業生、
全く油断出来ぬ。

などと考えていると、\EDの四人が会場へ入ってきた。
姿は完全に中学生であるが、彼らは既に大学院生を屠っている。
盛運尽きた俺が屠られるのは彼らになるのだろうか。
いや俺たち\urandomだって、筑波大会では最年少Rubyコミッターが率いるチームを倒したんだ。
目指すは彼らより上などと、大学生にしては少々大人気ない目標を掲げた。
まあ、優勝候補ならば許されるだろう。

彼らの島は我々の島の左隣り、彼らの動きは俺には見える。

