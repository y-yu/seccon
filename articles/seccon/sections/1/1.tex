俺とおしろ\footnote{情報科学類二年次、\WORD民。}はTXつくば駅の改札前で、
おーぴー達と合流した。
今回の会場は北千住にある東京電気大学らしいので、我々はTXで四十分程かけて向うことになる。
おーぴーはタイヤの付いたキャリーバッグと、モニターを装備していた。

列車に乗り込み、俺はKindleで本を読みながらも今日の大会のことをダラダラと考えている。
今日の全国大会はつくば大会などで行われたCTFとはシステムが違う。
つくばなどで行われた大会は、問題が用意され、その問題に解答する度にポイントが入る、
本当に\ruby{旗とりゲーム}{Capture\hfil The\hfil Flag}であったが、今回は違う。
今回行われるのは攻防戦。
\DEFCON\footnote{ラスベガスで行われる国際的なCTF。}などと同じようなものになるだろうという
のがおーぴーの読みだ。
ちなみに大会実行委員会から提示されたルールは、

\begin{itemize}
	\item 指定のターゲットサーバを攻略し、また他チームの攻撃から守る攻防戦を行う。
	\item ターゲットサーバを攻略し得られる「キーワード」を、スコアサーバにサブミットしたチームに、点数を与える。
	\item ターゲットサーバを管理していることを示す「フラッグワード」を、所定のWebページに示していた場合、そのチームに点数を与える。
\end{itemize}

という三つのみで、正直何が何やら分からない。
まあ、恐らく何が何やら分かっていないのは他チームも同じだろうし、
こちらは\DEFCON参加者であるおーぴーを擁している分、他チームよりも有利だろう。
いずれにしろ、つくば大会ではただ立っている旗を取れば良いだけであったが、
今回は取った旗を防御する必要があるのであろうというのは、なんとなく想像がつく。

自分達の旗を守り、かつ敵の旗を倒す。
さて、どうなることやら。
