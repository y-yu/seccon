つくば大会\footnote{2012年5月に行われた大会、{\WORD} 23号に掲載された。}は、本当に上手くいった。
もはや結果は覚えていないが、俺はSQLインジェクションの問題に貢献出来た。
横浜大会\footnote{%
つくば大会の次の次に行なわれた大会で、2012年12月に行われた。
\urandomも参加し成績は四位であった。}%
においても、SHA3\footnote{ハッシュ関数アルゴリズムのこと。}を用いた問題を解いて貢献することが出来た。

チーム\urandomはつくば大会二位の成績によって、今回の{\SECCON} CTF全国大会\footnote{2013年2月に行われた大会。}%
へ出場することとなった。
しかし、俺の盛運に翳りがないはずもない。
それでも俺が参加するのは、義理なのかもしれない。
つくば大会の時にはあれ程狂っていたものの、
最近は抗不安薬の服用によって幾分か病魔の発現は抑えられているので、
こんな精神を磨り減らすだけの戦いに挑むのは
もはやおーぴー\footnote{情報科学類二年次学生、\urandomのリーダー。CTFマニア。}への義理かもしれん。

まあ、いずれにしても俺はまた精神を犠牲にした戦いへ参加してしまった。
今度こそ壊滅的な敗北を喫し破滅するかもしれない。
