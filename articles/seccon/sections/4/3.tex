沈黙した\urandomに対して、\Marsで一気に加速した\ZxZが迫る。
それは飛ぶ鳥を落とす勢いで、\Jupiterで一瞬加速した\mofuppを抜き去り、
昨日の時点で800点あった点差を一気に400点まで詰める。

一騎打ちかと思いきや、サイレンが鳴り\ifconfigが\Venusのキーワードをコミットした。
うっ、\Venusは確かおーぴーがARMがどうのといっていた問題だったはず。
\ifconfigはそういう低レイヤーに関する専門家がいる。
あるところに情報科学類の学生が三人いた、一人はRubyを作り、
残りの二人はCPUなどを作ったという話をどこかで聞いたことがある。
この話に出てくる人間、
未踏\footnote{情報処理推進機構(IPA)が主催する、未踏IT人材発掘・育成事業のこと。}%
のスーパークリエータ\footnote{未踏採択者の中から、特に優秀であると評価された開発者に与えられる称号のこと。}%
になった人間が\ifconfigにはいる。
それになんといっても\ifconfigはつくば大会で我々を抜いて優勝した。
もし\VenusのFLAGに到達されたら不味い。

もはや俺は、敵チームに呪いをかける程度の能力しかなくなってしまったようだ。
こんなことをしていては運気が逃げる。
くそ、こうなったら\ifconfigに\Marsで俺が試した手の内をあかして協力を求めるのはどうだ。
\ifconfigには優れたペンテスターもいる。
俺も出来ることをやったつもりだが、もしかしたら見落しがあるかもしれん。
彼と協力すれば、あるいは解決出来るかも。
\ifconfigも盛り返しつつあるとはいえ、FLAGに届かないようではこのまま撃沈してしまうはず。
だったら、例え\urandomを首位にする結果になるにしても、
自身がなるべく良い結果になろうということで、利害が一致する可能性が------。

しかし、またサイレン。
\ifconfigが\Venusに二つ目のキーワードをコミットする。
ううっ、駄目だ。
この様子ではとても協力出来ない。
俺の話した内容から\Marsへの決定的なヒントを得て、
\ifconfigまでもが\Marsを占領したら目も当てられない。
\Venusは何時\ifconfigに占領されるかも分からぬ故、
二つも同時に占領されてはそのまま逆転もありえる。

するとまた長いサイレンが鳴り、今度は\wasamusumeが\NeptuneのFLAGを取得した。
\Neptuneはおしろ曰く、8080番ポートから壊れたパケットが飛んで来るとかいう問題で、
どうもあれはTCPではなくてUDPなのではなかろうかなどと言っていた覚えがある。
ただ、\wasamusume以外にキーワードをコミットしているチームすらいない所を考えると、
もしかしたらまたWindows 8みたいに、
マニアックなMicrosoft系の何かを使っているのかもしれない。

とりあえず、現在単独でFLAGを得ている\ZxZと\wasamusumeは極めて不味い。
まあどちらにしても残り時間から考えて、
仮にこのまま単独でFLAGを得続けていたとしても、
恐らく我々に追いつく前に時間切れといった感じではある。
しかしもしキーワードをいくつか追加されればそれで逆転もありえる。
よって俺はなんとしても\Marsを取り、
\ZxZの得点源を排除せねばならない。

くそったれ!
キーボードを叩き付ける。
しかし、もはや\Marsを攻める策がない。
その時何かのキーが押されたのか画面が遷移。

\begin{screen}
\textbf{cgi-lib.pl: Unknown Content-type: }
\end{screen}

ああ、なんだこのエラーは?
もしかして、この謎のライブラリ\ctt{cgi-lib.pl}に何かの脆弱性があるってことか。
一応、存在確認はしておくか。
\url{http://172.16.8.130/cgi-bin/cgi-lib.pl}にアクセス。

\begin{itembox}[c]{\textbf{Internal Server Error}}
The server encountered an internal error or misconfiguration and 
was unable to complete your request.
\end{itembox}

500、ああ、やっと\Marsで初めてエラーを見られた。
いや感動している場合ではない。
もしかしてこの\ctt{message.cgi}は全て囮だったのか。
いや囮というのは見え見えのOSコマンドインジェクションのことだ。
あれは何かあると思わせておくだけのデコイで、
内部で呼び出している\ctt{cgi-lib.pl}に脆弱性があり、
だから皆キーワードのコミットすら出来ていなんじゃないのか。
俺だって、このわけの分からない\ctt{cgi-lib.pl}なんて物体、
今キーボードを適当に叩いたから出現したくらいで普通では考えない。
並大抵のことではこの\ctt{cgi-lib.pl}に辿りつけないはず。

ならば、早速\ctt{cgi-lib.pl}の中身を調べなけけばなるまい。
まあこの手のプログラムは、
Webに落ちているのをそのまま流用したという可能性がある。
つまり、\ctt{cgi-lib.pl}はWebのどっかに落ちているのではなかろうか。
直ちに検索開始。
Googleで調べると、直ちに配布元が特定出来た。
ソース\footnote{\url{http://cgi-lib.berkeley.edu/2.18/cgi-lib.pl.txt}}もある。
がこれは、結構デカいな。

この中にありえそうな脆弱性といえば、なんだ?
これ、本当に正しいのか?
このライブラリ、\ctt{message.cgi}の中でどう呼び出されているのかも分からんというのに、
脆弱性を探せなどというのは無理があるんじゃないか。
やはり、\ctt{message.cgi}にはOSコマンドインジェクションがあるものの、
それは\ZxZの妨害工作で見えなくなっていただけという説に回帰。

